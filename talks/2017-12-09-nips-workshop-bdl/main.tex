\documentclass[10pt,
               xcolor={usenames,dvipsnames},
               hyperref={colorlinks,linktoc=all,citecolor=Plum,linkcolor=MidnightBlue,urlcolor=MidnightBlue},noamssymb]{beamer}
\usepackage{etex}
\usepackage[T1]{fontenc}
\usepackage[bitstream-charter]{mathdesign}
\usepackage[scaled=0.92]{PTSans}
\renewcommand*\ttdefault{lmvtt}
\usepackage{exscale}

\usepackage[final,expansion=alltext]{microtype}
\usepackage[english]{babel}
\usepackage{ragged2e}

\usepackage{amsmath}
\usepackage{graphicx}                               % graphics
\usepackage{booktabs, array}                        % nice tables
\usepackage[algoruled]{algorithm2e}                 % pseudocode
\usepackage[numbers,sort]{natbib}                   % better bibliography
\usepackage{cleveref}                               % the best package ever
\usepackage[acronym,smallcaps,nowarn]{glossaries}   % second best package ever

% COLORS
\newcommand{\gray}[1]{\textcolor{black!60}{#1}}
\newcommand{\blu}[1]{\textcolor{jblue}{#1}}

\newacronym{ELBO}{elbo}{evidence lower bound}
\newacronym{KL}{kl}{Kullback-Leibler}

\newacronym{DEF}{def}{deep exponential family}
\newacronym{DLGM}{dlgm}{deep latent Gaussian model}
\newacronym{DRAW}{draw}{Deep Recurrent Attentive Writer}

\newacronym{MF}{mf}{mean-field}
\newacronym{VGP}{vgp}{variational Gaussian process}
\newacronym{BBVI}{bbvi}{black box variational inference}
\newacronym{ADVI}{advi}{automatic differentiation variational inference}
\newacronym{NF}{nf}{normalizing flows}
\newacronym{CVI}{cvi}{copula variational inference}
\newacronym{VAE}{vae}{variational autoencoder}
\newacronym{IWAE}{iwae}{importance weighted autoencoder}
\newacronym{NVIL}{nvil}{neural variational inference}
\newacronym{MIXTURE}{mixture}{}
\newacronym{DSVI}{dsvi}{}

\newacronym{VI}{vi}{variational inference}
\newacronym{EP}{ep}{expectation propagation}

\newacronym{LS}{ls}{Langevin-Stein}
\newacronym{OPVI}{opvi}{operator variational inference}

\DeclareRobustCommand{\mb}[1]{\ensuremath{\boldsymbol{\mathbf{#1}}}}
\DeclareRobustCommand{\mbf}[1]{\ensuremath{\textbf{#1}}}
\DeclareMathOperator*{\argmax}{arg\,max}
\DeclareMathOperator*{\argmin}{arg\,min}


\newcommand{\KL}[2]{\ensuremath{\textrm{KL}\left(#1\;\|\;#2\right)}}

\renewcommand{\mid}{~\vert~}

\newcommand{\mbw}{\mb{w}}
\newcommand{\mbW}{\mb{W}}

\newcommand{\mbx}{\mb{x}}
\newcommand{\mbX}{\mbf{X}}

\newcommand{\mby}{\mb{y}}
\newcommand{\mbY}{\mbf{Y}}

\newcommand{\mbz}{\mb{z}}
\newcommand{\mbZ}{\mb{Z}}

\newcommand{\mbI}{\mbf{I}}
\newcommand{\mbone}{\mbf{1}}

\newcommand{\mbL}{\mbf{L}}

\newcommand{\mbtheta}{\mb{\theta}}
\newcommand{\mbTheta}{\mb{\Theta}}
\newcommand{\mbomega}{\mb{\omega}}
\newcommand{\mbOmega}{\mb{\Omega}}
\newcommand{\mbsigma}{\mb{\sigma}}
\newcommand{\mbSigma}{\mb{\Sigma}}
\newcommand{\mbphi}{\mb{\phi}}
\newcommand{\mbPhi}{\boldsymbol{\Phi}}
% \newcommand{\mblambda}{\mb{\lambda}}

\newcommand{\mbalpha}{\mb{\alpha}}
\newcommand{\mbbeta}{\mb{\beta}}
\newcommand{\mbgamma}{\mb{\gamma}}
\newcommand{\mbeta}{\mb{\eta}}
\newcommand{\mbmu}{\mb{\mu}}
\newcommand{\mbrho}{\mb{\rho}}
\newcommand{\mblambda}{\mb{\lambda}}
\newcommand{\mbzeta}{\mb{\zeta}}

\newcommand\dif{\mathop{}\!\mathrm{d}}
\newcommand{\diag}{\textrm{diag}}
\newcommand{\supp}{\textrm{supp}}

\newcommand{\E}{\mathbb{E}}
\newcommand{\bbH}{\mathbb{H}}
\newcommand{\Var}{\mathbb{V}\textrm{ar}}

\newcommand{\bbN}{\mathbb{N}}
\newcommand{\bbZ}{\mathbb{Z}}
\newcommand{\bbR}{\mathbb{R}}
\newcommand{\bbS}{\mathbb{S}}

\newcommand{\cL}{\mathcal{L}}

\newcommand{\cN}{\mathcal{N}}
\newcommand{\cT}{\mathcal{T}}
\newcommand{\Gam}{\textrm{Gam}}
\newcommand{\InvGam}{\textrm{InvGam}}

\usepackage{tikz}
\usetikzlibrary{bayesnet}

\pgfdeclarelayer{edgelayer}
\pgfdeclarelayer{nodelayer}
\pgfsetlayers{edgelayer,nodelayer,main}

\definecolor{hexcolor0xbfbfbf}{rgb}{0.749,0.749,0.749}

\tikzset{>=latex}
\tikzstyle{none}   = [inner sep=0pt]
\tikzstyle{line}   = [-,
                      thick,
                      shorten <=1pt,
                      shorten >=1pt]
\tikzstyle{arrow}  = [->,
                      thick,
                      shorten <=1pt,
                      shorten >=1pt]
\tikzstyle{ardash} = [dashed,
                      ->,
                      thick,
                      shorten <=1pt,
                      shorten >=1pt]

\tikzstyle{empty}=[
                   circle,
                   opacity=0.0,
                   text opacity=1.0,
                   inner sep=0pt
                  ]

\tikzstyle{box}=[
                 rectangle,
                 fill=White,
                 thick,
                 draw=Black,
                 inner sep=7pt
                ]

\tikzstyle{filled}=[
                    circle,
                    thick,
                    fill=hexcolor0xbfbfbf,
                    draw=Black
                   ]

\tikzstyle{hollow}=[
                    circle,
                    thick,
                    fill=White,
                    draw=Black
                   ]

\tikzstyle{param}=[
                   rectangle,
                   fill=Black,
                   draw=Black,
                   inner sep=0pt,
                   minimum width=4pt,
                   minimum height=4pt
                  ]

\tikzstyle{paramhollow}=[
                         rectangle,
                         thick,
                         fill=White,
                         draw=Black,
                         inner sep=0pt,
                         minimum
                         width=4pt,
                         minimum height=4pt
                        ]

\usepackage{pgfplots}                               % PGFPLOTS baby!
\pgfplotsset{compat=newest}
\pgfplotsset{plot coordinates/math parser=false}
% \usepgfplotslibrary{statistics}


\usepackage{subfigure}
\definecolor{light}{RGB}{199, 153, 199}
\definecolor{dark}{RGB}{143, 39, 143}
\definecolor{gray80}{gray}{0.8}

\usepackage{natbib}
\setbeamertemplate{navigation symbols}{}
\setbeamertemplate{itemize items}[circle]
\setbeamercolor{itemize item}{fg=black!67}
\setbeamercolor*{enumerate item}{fg=black!67}
\setbeamercolor*{enumerate subitem}{fg=black!67}
\setbeamercolor*{enumerate subsubitem}{fg=black!67}

\definecolor{charcoal}{HTML}{222222}
\definecolor{snow}{HTML}{F9F9F9}

\setbeamercolor{background canvas}{bg=white}
\setbeamercolor{normal text}{fg=charcoal}
\setbeamercolor{structure}{fg=charcoal}

\newenvironment{changemargin}[1]{
  \begin{list}{}{
    \setlength{\topsep}{0pt}
    \setlength{\leftmargin}{#1}
    \setlength{\rightmargin}{#1}
    \setlength{\listparindent}{\parindent}
    \setlength{\itemindent}{\parindent}
    \setlength{\parsep}{\parskip}
  }
  \item[]}{\end{list}}

\title{}
\begin{document}

\begin{frame}[plain,t]
\begin{tikzpicture}[remember picture,overlay]
  \node [xshift=0.50cm, yshift=-3.00cm, anchor=north west] at (current page.north west) {
    \begin{tabular}{l}
    {\Large\bf Why Aren't You Using}\\[2ex]
    {\Large\bf Probabilistic Programming?}\\[2ex]
    {\large }\\[4ex]
    Dustin Tran\\
    Columbia University \\[4ex]
    \end{tabular}
  };
  \node [xshift=-1.50cm, yshift=3.00cm, anchor=mid east] at (current page.south
  east) {
\includegraphics[width=0.25\textwidth]{img/edward.png}
  };
  \node [xshift=-0.25cm, yshift=0.5cm, anchor=mid east] at (current page.south
  east) {
    \includegraphics[width=0.22\textwidth]{img/columbia.pdf}
  };
\end{tikzpicture}
\end{frame}

% + why as in current problems in ppls
% + what the languages have led to and edward
% + current limitaitons, and why you shouldn't use probabilistic
% programming yet

\begin{frame}[plain]
\footnotesize
\begin{center}
\begin{tabular}{ccccc}
\includegraphics[width=18mm]{img/alp.png} &
\includegraphics[width=18mm]{img/adji.jpg} &
\includegraphics[width=18mm]{img/dave.jpg} &
\includegraphics[width=18mm]{img/dawen.jpg} &
\includegraphics[width=18mm]{img/eugene.jpg} \\
Alp Kucukelbir & Adji Dieng & Dave Moore & Dawen Liang & Eugene Brevdo \\
\end{tabular}

\vspace{-2ex}

\begin{tabular}{ccccc}
\includegraphics[width=18mm]{img/ian.jpg} &
\includegraphics[width=18mm]{img/josh.jpg} &
\includegraphics[width=18mm]{img/maja.png} &
\includegraphics[width=18mm]{img/brian.png} &
\\
% \includegraphics[width=18mm]{img/srinivas.png} \\
Ian Langmore & Josh Dillon & Maja Rudolph & Brian Patton & Srinivas Vasudevan \\
\end{tabular}

\vspace{-2ex}

\begin{tabular}{ccccc}
\includegraphics[width=18mm]{img/alex.jpg} &
\includegraphics[width=20mm]{img/matt.jpg} &
\includegraphics[width=18mm]{img/blei.jpg} &
\includegraphics[width=18mm]{img/kevin.jpg} &
\includegraphics[width=18mm]{img/rif.png}\\
Alex Alemi & Matt Hoffman & David Blei & Kevin Murphy & Rif Saurous\\
\end{tabular}
\end{center}
\end{frame}

\begin{frame}
\frametitle{What is probabilistic programming?}
\textbf{Probabilistic programs reify models from mathematics to
physical objects.}
\begin{itemize}
\vspace{-2ex}
\item
Each model is equipped with memory (``bits'',
floating point, storage) and computation
(``flops'', scalability, communication).
% from Gauss and Fisher to Turing and Church
\end{itemize}
\textbf{Anything you do lives in the world of probabilistic programming.}
\begin{itemize}
\item
Any computable model.
\item
Any computable inference algorithm.
\item
Any computable application.
\end{itemize}
\end{frame}

\begin{frame}
\begin{center}
{\Huge
\textit{\textbf{Simulation hypothesis.} \\
``The universe is a simulation from a computer program.''
}
\\[2ex]
{\Large
(Zuse, Schmidhuber, Bostrom, Musk)
}}
\end{center}
\end{frame}
% dogmatic puritanical computability to good software desgin with modular code and rich abstractions encapsulation ... to this we deceloped edward in fsct edward was originally ...

\begin{frame}
\vspace{-20ex}
\begin{center}
\includegraphics[width=1.0\textwidth]{img/query.pdf}
\end{center}
\begin{tikzpicture}[remember picture,overlay]
  \node [xshift=-6.25cm, yshift=0.4cm, anchor=south west] at (current
  page.south east) {
\gray{\small [Tenenbaum+Mansinghka NIPS 2017 tutorial]}
  };
\end{tikzpicture}
\end{frame}

\begin{frame}
\begin{tikzpicture}[remember picture,overlay]
  \node [xshift=-0.75cm, yshift=0.8cm, anchor=north west] at (current
  page.north west) {
\includegraphics[width=0.8\textwidth]{img/blog.png}
  };
  \node [xshift=-0.5cm, yshift=-3.0cm, anchor=north west] at (current
  page.north west) {
\includegraphics[width=0.8\textwidth]{img/repo-1.png}
  };
  \node [xshift=-0.5cm, yshift=-6.0cm, anchor=north west] at (current
  page.north west) {
\includegraphics[width=0.8\textwidth]{img/repo-2.png}
  };
  \node [xshift=5.0cm, yshift=0.1cm, anchor=north west] at (current
  page.north west) {
\includegraphics[width=0.8\textwidth]{img/repo-3.png}
  };
  \node [xshift=5.5cm, yshift=-3.0cm, anchor=north west] at (current
  page.north west) {
\includegraphics[width=0.8\textwidth]{img/repo-4.png}
  };
  \node [xshift=5.5cm, yshift=-6.0cm, anchor=north west] at (current
  page.north west) {
\includegraphics[width=0.8\textwidth]{img/repo-5.png}
  };
\end{tikzpicture}
\end{frame}

% \begin{frame}
% \begin{tikzpicture}[remember picture,overlay]
%   \node [xshift=-0.75cm, yshift=0.8cm, anchor=north west] at (current
%   page.north west) {
% \includegraphics[width=0.8\textwidth]{img/blog.png}
%   };
%   \node [xshift=-0.5cm, yshift=-3.0cm, anchor=north west] at (current
%   page.north west) {
% \includegraphics[width=0.8\textwidth]{img/paper-1.png}
%   };
%   \node [xshift=-0.5cm, yshift=-6.0cm, anchor=north west] at (current
%   page.north west) {
% \includegraphics[width=0.8\textwidth]{img/neural-architecture.png}
%   };
%   \node [xshift=5.0cm, yshift=0.1cm, anchor=north west] at (current
%   page.north west) {
% \includegraphics[width=0.8\textwidth]{img/paper-2.png}
%   };
%   \node [xshift=5.5cm, yshift=-3.0cm, anchor=north west] at (current
%   page.north west) {
% \includegraphics[width=0.8\textwidth]{img/paper-3.png}
%   };
%   \node [xshift=5.5cm, yshift=-6.0cm, anchor=north west] at (current
%   page.north west) {
% \includegraphics[width=0.8\textwidth]{img/paper-4.png}
%   };
% \end{tikzpicture}
% % replace with
% % stochastic depth
% % gating
% \end{frame}

\begin{frame}[t]
\frametitle{The Myth of Probabilistic Programming}
\vspace{16ex}

\begin{center}
{\Large
\bf
Programming is infeasible if a core operation \\[1.25ex]
in the language is
NP-hard.
}
% (or worse!)
\end{center}

\vspace{16ex}
For high-dimensional problems + modern probabilistic models, we
haven't solved automated inference.
\end{frame}

\begin{frame}
\begin{center}
\vspace{-2.5ex}
\includegraphics[width=1.0\textwidth]{img/github.png}
\\[-1.5ex]
\includegraphics[width=1.0\textwidth]{img/forum.png}
\\[-3ex]
\includegraphics[width=1.0\textwidth]{img/gitter.png}
\\[2ex]
\end{center}
\text{
We have an active community of several thousand users \& many
contributors.
}
\end{frame}

\begin{frame}
\frametitle{Language: Computational Graphs w/ Random Variables}
Edward's language augments computational graphs with an abstraction
for random variables.
Each random variable $\mbx$ is associated to a tensor $\mbx^*$,
$\mbx^*\sim p(\mbx\g\theta^*)$.

\vspace{-1.0ex}
\includegraphics[height=0.20\textwidth]{img/random_variables.png}

Unlike \texttt{tf.Tensor}s, \texttt{ed.RandomVariable}s
carry an explicit density with methods
such as \texttt{log\_prob()} and \texttt{sample()}.

For implementation, we wrap all TensorFlow Distributions and call
\texttt{sample} to produce the associated tensor.
\begin{tikzpicture}[remember picture,overlay]
  \node [xshift=-2.2cm, yshift=0.4cm, anchor=south west] at (current
  page.south east) {
\gray{\small [Tran+ 2017]}
  };
\end{tikzpicture}
\end{frame}

\begin{frame}
\frametitle{Language Example}
\begin{center}
\includegraphics[width=1.05\textwidth]{img/ssm-program.png}
\end{center}

\vspace{1ex}
State space model for sequences
$\mathbf{x}=[\mathbf{x}_1,\ldots,\mathbf{x}_T]\in\mathbb{R}^{T\times D}$.
\vspace{2ex}

Edward's language enables a \emph{calculus} on random variables.
\begin{tikzpicture}[remember picture,overlay]
  \node [xshift=-2.25cm, yshift=0.4cm, anchor=south west] at (current
  page.south east) {
\gray{\small [Dillon+ 2017]}
  };
\end{tikzpicture}
\end{frame}

\begin{frame}
\frametitle{Inference as Stochastic Graph Optimization}

\begin{center}
\includegraphics[width=0.7\textwidth]{img/inference-graph.png}
\end{center}

All \texttt{Inference} has (at least) two inputs: \\
\begin{enumerate}
\vspace{-3ex}
\item
\red{red} aligns latent variables and posterior approximations;
\item
\blue{blue} aligns observed variables and realizations.
\end{enumerate}

\begin{center}
\vspace{-2.0ex}
\includegraphics[height=0.05\textheight]{img/inference.png}
\end{center}

\texttt{Inference} has class methods to finely control the algorithm.
Edward is as fast as handwritten TensorFlow at runtime.
\begin{tikzpicture}[remember picture,overlay]
  \node [xshift=-3.5cm, yshift=0.4cm, anchor=south west] at (current
  page.south east) {
\small \url{edwardlib.org/api}
  };
\end{tikzpicture}
\end{frame}

\begin{frame}
\frametitle{Composable \& Hybrid Inference}

\begin{center}
\vspace{-2ex}
\includegraphics[width=1.0\textwidth]{img/em.png}
\vspace{2ex}

\includegraphics[width=1.0\textwidth]{img/ep.png}
\end{center}

\begin{tikzpicture}[remember picture,overlay]
  \node [xshift=-9.2cm, yshift=0.4cm, anchor=south west] at (current
  page.south east) {
\gray{\small [Neal \& Hinton 1993; Minka 2001; Gelman+ 2017; Hasenclever+ 2015]}
  };
\end{tikzpicture}
\end{frame}

\begin{frame}
\frametitle{Non-Bayesian Inference}
\begin{center}
\vspace{-2ex}
\includegraphics[width=0.9\textwidth]{img/gan_example.png}
\end{center}
\begin{tikzpicture}[remember picture,overlay]
  \node [xshift=-4.5cm, yshift=0.4cm, anchor=south west] at (current
  page.south east) {
\gray{\small [Dayan+ 1995; Gutmann+ 2010]}
  };
\end{tikzpicture}
\end{frame}

\begin{frame}
\frametitle{Taxonomy of Inference}
\vspace{-1ex}
\begin{center}
\includegraphics[width=1.05\textwidth]{img/taxonomy.png}
\end{center}
\vspace{10ex}
Nodes are Edward classes. Arrows denote inheritance.
\begin{tikzpicture}[remember picture,overlay]
  \node [xshift=-3.5cm, yshift=0.4cm, anchor=south west] at (current
  page.south east) {
\small \url{edwardlib.org/api}
  };
\end{tikzpicture}
\end{frame}

\begin{frame}
\frametitle{Why You Shouldn't Use Probabilistic Programming}
\begin{itemize}
\item
\textbf{Object-oriented inference} has a high cognitive burden.
\item
Sometimes it's easier to \textbf{build the loss function} than it
is to build the program.
(e.g. autoregressive models)
\item
\textbf{Programmable inference is hard}. Matt and I spent a year on
covering use cases. But we didn't cover all of them.
\end{itemize}
\end{frame}

\begin{frame}
\begin{center}
{\Large\bf Why You Will Use Probabilistic Programming}
\end{center}
\end{frame}

\begin{frame}
\frametitle{Distributed, Compiled, Accelerated Systems}
\begin{center}
\vspace{-1.25ex}
\includegraphics[width=0.8\textwidth]{img/tpu-pods.png}
\end{center}

\vspace{3ex}
Probabilistic programming over multiple machines.
XLA compiler optimization and TPUs.
% Turing-complete meta language for inference.
More flexible programmable inference.
\end{frame}

\begin{frame}[t]
\frametitle{Dynamic Graphs}
\begin{center}
\vspace{-1.25ex}
\includegraphics[width=1.0\textwidth]{img/pyro.png}
\\[-8.75ex]
\includegraphics[width=1.0\textwidth]{img/probtorch.png}
\end{center}
\end{frame}

\begin{frame}
\frametitle{Distributions Backend}
\vspace{-3ex}
\begin{lstlisting}[language=python]
def pixelcnn_dist(params, x_shape=(32, 32, 3)):
  def _logit_func(features):
    # single autoregressive step on observed features
    logits = pixelcnn(features)
    return logits
  logit_template = tf.make_template("pixelcnn", _logit_func)
  make_dist = lambda x: tfd.Independent(tfd.Bernoulli(logit_template(x)))
  return tfd.Autoregressive(make_dist, tf.reduce_prod(x_shape)))

x = pixelcnn_dist()
loss = -tf.reduce_sum(x.log_prob(images))
train = tf.train.AdamOptimizer().minimize(loss)  # run for training
generate = x.sample()  # run for generation
\end{lstlisting}


\textbf{TensorFlow Distributions} consists of a large collection of
distributions. \texttt{Bijector} enable efficient, composable
manipulation of probability distributions.
% pixelcnn, autoregressive flows, reversible resnet

Pytorch PPLs are consolidating on a backend for distributions.
\begin{tikzpicture}[remember picture,overlay]
  \node [xshift=-2.25cm, yshift=0.4cm, anchor=south west] at (current
  page.south east) {
\gray{\small [Dillon+ 2017]}
  };
\end{tikzpicture}
\end{frame}

% doc, examples, tutorials. numericsl stability, etc maybe included as why

\begin{frame}
\frametitle{References}
\begin{center}
\includegraphics[width=0.3\textwidth]{img/edward.png}
\\
\large \url{edwardlib.org}
\end{center}
\vspace{1ex}

\begin{itemize}
\item
Edward: A library for probabilistic modeling, inference, and
criticism. \\
\gray{arXiv preprint arXiv:1610.09787, 2016.}
\item
Deep probabilistic programming. \\
\gray{International Conference on Learning Representations, 2017.}
\item
TensorFlow Distributions. \\
\gray{arXiv preprint arXiv:1711.10604, 2017.}
\end{itemize}
\end{frame}

\end{document}
