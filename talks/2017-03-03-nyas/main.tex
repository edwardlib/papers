%##############################################################################
% Preamble
%##############################################################################
\documentclass[10pt,
               xcolor={usenames,dvipsnames},
               hyperref={colorlinks,linktoc=all,citecolor=Plum,linkcolor=MidnightBlue,urlcolor=MidnightBlue},noamssymb]{beamer}
\usepackage{etex}
\usepackage[T1]{fontenc}
\usepackage[bitstream-charter]{mathdesign}
\usepackage[scaled=0.92]{PTSans}
\renewcommand*\ttdefault{lmvtt}
\usepackage{exscale}

\usepackage[final,expansion=alltext]{microtype}
\usepackage[english]{babel}
\usepackage{ragged2e}

\usepackage{amsmath}
\usepackage{graphicx}                               % graphics
\usepackage{booktabs, array}                        % nice tables
\usepackage[algoruled]{algorithm2e}                 % pseudocode
\usepackage[numbers,sort]{natbib}                   % better bibliography
\usepackage{cleveref}                               % the best package ever
\usepackage[acronym,smallcaps,nowarn]{glossaries}   % second best package ever

% COLORS
\newcommand{\gray}[1]{\textcolor{black!60}{#1}}
\newcommand{\blu}[1]{\textcolor{jblue}{#1}}

\newacronym{ELBO}{elbo}{evidence lower bound}
\newacronym{KL}{kl}{Kullback-Leibler}

\newacronym{DEF}{def}{deep exponential family}
\newacronym{DLGM}{dlgm}{deep latent Gaussian model}
\newacronym{DRAW}{draw}{Deep Recurrent Attentive Writer}

\newacronym{MF}{mf}{mean-field}
\newacronym{VGP}{vgp}{variational Gaussian process}
\newacronym{BBVI}{bbvi}{black box variational inference}
\newacronym{ADVI}{advi}{automatic differentiation variational inference}
\newacronym{NF}{nf}{normalizing flows}
\newacronym{CVI}{cvi}{copula variational inference}
\newacronym{VAE}{vae}{variational autoencoder}
\newacronym{IWAE}{iwae}{importance weighted autoencoder}
\newacronym{NVIL}{nvil}{neural variational inference}
\newacronym{MIXTURE}{mixture}{}
\newacronym{DSVI}{dsvi}{}

\newacronym{VI}{vi}{variational inference}
\newacronym{EP}{ep}{expectation propagation}

\newacronym{LS}{ls}{Langevin-Stein}
\newacronym{OPVI}{opvi}{operator variational inference}

\DeclareRobustCommand{\mb}[1]{\ensuremath{\boldsymbol{\mathbf{#1}}}}
\DeclareRobustCommand{\mbf}[1]{\ensuremath{\textbf{#1}}}
\DeclareMathOperator*{\argmax}{arg\,max}
\DeclareMathOperator*{\argmin}{arg\,min}


\newcommand{\KL}[2]{\ensuremath{\textrm{KL}\left(#1\;\|\;#2\right)}}

\renewcommand{\mid}{~\vert~}

\newcommand{\mbw}{\mb{w}}
\newcommand{\mbW}{\mb{W}}

\newcommand{\mbx}{\mb{x}}
\newcommand{\mbX}{\mbf{X}}

\newcommand{\mby}{\mb{y}}
\newcommand{\mbY}{\mbf{Y}}

\newcommand{\mbz}{\mb{z}}
\newcommand{\mbZ}{\mb{Z}}

\newcommand{\mbI}{\mbf{I}}
\newcommand{\mbone}{\mbf{1}}

\newcommand{\mbL}{\mbf{L}}

\newcommand{\mbtheta}{\mb{\theta}}
\newcommand{\mbTheta}{\mb{\Theta}}
\newcommand{\mbomega}{\mb{\omega}}
\newcommand{\mbOmega}{\mb{\Omega}}
\newcommand{\mbsigma}{\mb{\sigma}}
\newcommand{\mbSigma}{\mb{\Sigma}}
\newcommand{\mbphi}{\mb{\phi}}
\newcommand{\mbPhi}{\boldsymbol{\Phi}}
% \newcommand{\mblambda}{\mb{\lambda}}

\newcommand{\mbalpha}{\mb{\alpha}}
\newcommand{\mbbeta}{\mb{\beta}}
\newcommand{\mbgamma}{\mb{\gamma}}
\newcommand{\mbeta}{\mb{\eta}}
\newcommand{\mbmu}{\mb{\mu}}
\newcommand{\mbrho}{\mb{\rho}}
\newcommand{\mblambda}{\mb{\lambda}}
\newcommand{\mbzeta}{\mb{\zeta}}

\newcommand\dif{\mathop{}\!\mathrm{d}}
\newcommand{\diag}{\textrm{diag}}
\newcommand{\supp}{\textrm{supp}}

\newcommand{\E}{\mathbb{E}}
\newcommand{\bbH}{\mathbb{H}}
\newcommand{\Var}{\mathbb{V}\textrm{ar}}

\newcommand{\bbN}{\mathbb{N}}
\newcommand{\bbZ}{\mathbb{Z}}
\newcommand{\bbR}{\mathbb{R}}
\newcommand{\bbS}{\mathbb{S}}

\newcommand{\cL}{\mathcal{L}}

\newcommand{\cN}{\mathcal{N}}
\newcommand{\cT}{\mathcal{T}}
\newcommand{\Gam}{\textrm{Gam}}
\newcommand{\InvGam}{\textrm{InvGam}}

\usepackage{tikz}
\usetikzlibrary{bayesnet}

\pgfdeclarelayer{edgelayer}
\pgfdeclarelayer{nodelayer}
\pgfsetlayers{edgelayer,nodelayer,main}

\definecolor{hexcolor0xbfbfbf}{rgb}{0.749,0.749,0.749}

\tikzset{>=latex}
\tikzstyle{none}   = [inner sep=0pt]
\tikzstyle{line}   = [-,
                      thick,
                      shorten <=1pt,
                      shorten >=1pt]
\tikzstyle{arrow}  = [->,
                      thick,
                      shorten <=1pt,
                      shorten >=1pt]
\tikzstyle{ardash} = [dashed,
                      ->,
                      thick,
                      shorten <=1pt,
                      shorten >=1pt]

\tikzstyle{empty}=[
                   circle,
                   opacity=0.0,
                   text opacity=1.0,
                   inner sep=0pt
                  ]

\tikzstyle{box}=[
                 rectangle,
                 fill=White,
                 thick,
                 draw=Black,
                 inner sep=7pt
                ]

\tikzstyle{filled}=[
                    circle,
                    thick,
                    fill=hexcolor0xbfbfbf,
                    draw=Black
                   ]

\tikzstyle{hollow}=[
                    circle,
                    thick,
                    fill=White,
                    draw=Black
                   ]

\tikzstyle{param}=[
                   rectangle,
                   fill=Black,
                   draw=Black,
                   inner sep=0pt,
                   minimum width=4pt,
                   minimum height=4pt
                  ]

\tikzstyle{paramhollow}=[
                         rectangle,
                         thick,
                         fill=White,
                         draw=Black,
                         inner sep=0pt,
                         minimum
                         width=4pt,
                         minimum height=4pt
                        ]

\usepackage{pgfplots}                               % PGFPLOTS baby!
\pgfplotsset{compat=newest}
\pgfplotsset{plot coordinates/math parser=false}
% \usepgfplotslibrary{statistics}


\usepackage{subfigure}
\definecolor{light}{RGB}{199, 153, 199}
\definecolor{dark}{RGB}{143, 39, 143}
\definecolor{gray80}{gray}{0.8}

\usepackage{natbib}
\setbeamertemplate{navigation symbols}{}
\setbeamertemplate{itemize items}[circle]
\setbeamercolor{itemize item}{fg=black!67}
\setbeamercolor*{enumerate item}{fg=black!67}
\setbeamercolor*{enumerate subitem}{fg=black!67}
\setbeamercolor*{enumerate subsubitem}{fg=black!67}

\definecolor{charcoal}{HTML}{222222}
\definecolor{snow}{HTML}{F9F9F9}

\setbeamercolor{background canvas}{bg=white}
\setbeamercolor{normal text}{fg=charcoal}
\setbeamercolor{structure}{fg=charcoal}

\newenvironment{changemargin}[1]{
  \begin{list}{}{
    \setlength{\topsep}{0pt}
    \setlength{\leftmargin}{#1}
    \setlength{\rightmargin}{#1}
    \setlength{\listparindent}{\parindent}
    \setlength{\itemindent}{\parindent}
    \setlength{\parsep}{\parskip}
  }
  \item[]}{\end{list}}

\title{}
\begin{document}

\begin{frame}[plain,t]
\begin{tikzpicture}[remember picture,overlay]
  \node [xshift=0.50cm, yshift=-3.00cm, anchor=north west] at (current page.north west) {
    \begin{tabular}{l}
    {\Large\bf Edward: A library for probabilistic modeling,}\\[1ex]
    {\Large\bf inference, and criticism}\\[2ex]
    {\large }\\[4ex]
    Dustin Tran, David M. Blei\\
    Columbia University\\[4ex]
    Matt Hoffman, Rif A. Saurous, Eugene Brevdo, \\
    Kevin Murphy \\
    Google Brain
    \end{tabular}
  };
  \node [xshift=-1.50cm, yshift=3.00cm, anchor=mid east] at (current page.south
  east) {
\includegraphics[width=0.25\textwidth]{img/edward.png}
  };
  \node [xshift=-1.30cm, yshift=2.45cm, anchor=mid east] at (current page.south
  east) {
\large \url{edwardlib.org}
  };
\end{tikzpicture}
\end{frame}

\begin{frame}
\frametitle{George E.P. Box (1919 - 2013)}
\begin{columns}
\begin{column}{0.5\textwidth}
    \begin{center}
     \includegraphics[width=\columnwidth]{img/box.jpg}
     \end{center}
\end{column}
\begin{column}{0.5\textwidth}
An iterative process for science:
\\[1ex]
\begin{enumerate}
\item Build a model of the science
\\[1ex]
\item Infer the model given data
\\[1ex]
\item Criticize the model given data
\end{enumerate}
\end{column}
\end{columns}
\begin{tikzpicture}[remember picture,overlay]
  \node [xshift=-9cm, yshift=0.4cm, anchor=south west] at (current
  page.south east) {
\gray{(Box \& Hunter 1962, 1965; Box \& Hill 1967; Box 1976, 1980)}
  };
\end{tikzpicture}
\end{frame}

\begin{frame}
\frametitle{Box's Loop}
\begin{tikzpicture}[remember picture,overlay]
  \node [xshift=-1cm, yshift=-2.00cm, anchor=north west] at (current page.north west) {
\includegraphics[width=1.4\textwidth]{img/model_infer_criticize.png}
  };
  \node [xshift=3.4cm, yshift=-8.0cm, anchor=north west] at (current page.north west) {
Edward is a library designed around this loop.
  };
  \node [xshift=0cm, yshift=-5.50cm, anchor=north west] at (current page.north west) {
  };
  \node [xshift=-5.0cm, yshift=0.4cm, anchor=south west] at (current
  page.south east) {
\gray{(Box, 1976; Box, 1980; Blei, 2014)}
  };
\end{tikzpicture}
\end{frame}

\begin{frame}
\vspace{3ex}
\textbf{Edward} is a probabilistic programming language,
designed for fast experimentation and research.

\emph{Modeling}
\begin{itemize}
\item
Composable Turing-complete language of random variables.
\item
Examples:
Graphical models, neural networks, probabilistic programs.
\item
Many data types, tensor vectorization, broadcasting, 3rd party support.
\end{itemize}

\emph{Inference}
\begin{itemize}
\item
Composable language for hybrids, message passing, data subsampling.
\item
Examples:
Black box VI, Hamiltonian MC, stochastic
gradient MCMC.
\item
Infrastructure to develop your own algorithms.
\end{itemize}

\emph{Criticism}
\begin{itemize}
\item
Examples: Scoring rules, hypothesis tests, predictive checks.
\end{itemize}

\vspace{1ex}
Built on TensorFlow (features distributed computing, GPUs, autodiff).

\begin{tikzpicture}[remember picture,overlay]
  \node [xshift=-3cm, yshift=0.4cm, anchor=south west] at (current
  page.south east) {
\gray{(Tran et al., 2016)}
  };
\end{tikzpicture}
\end{frame}

\begin{frame}
\begin{center}
\vspace{-4ex}
\includegraphics[width=1.1\textwidth]{img/github.png}
\\
\includegraphics[width=1.0\textwidth]{img/gitter.png}
\\[3ex]
\end{center}
We have an active community of several hundred users. We have
many few-commit developers.
\end{frame}

\begin{frame}
\frametitle{Who is Using Edward?}
{\large Users}
\begin{enumerate}
\item
Machine learning enthusiasts, data scientists, business analysts \\
(\emph{ex. hierarchical GLMs, mixture models, MAP, MCMC, ...})
\item
Probabilistic graphical modeling community \\
(\emph{ex. latent Dirichlet allocation, variational inference, Gibbs})
\item
Bayesian deep learning community \\
(\emph{ex. deep generative models, Bayesian NNs, black box inference})
\end{enumerate}

{\large Developers}
\begin{enumerate}
\item
David Blei's group
\item
Google Brain
(\emph{in conception/design})
\item
Matt Hoffman (\emph{conjugacy}),
Emily Fox's group
(\emph{time series + SGMCMC}),
Justin Bayer (\emph{stochastic RNNs}),
John Pearson (\emph{neuroscience}),
a few Master's/Ph.D. students.
\item
Everyone is part-time. Collaboration continues to evolve.
\end{enumerate}
\end{frame}

\begin{frame}[plain,t]
\frametitle{Variational Auto-Encoder}
\begin{tikzpicture}[remember picture,overlay]
  \node [xshift=12.40cm, yshift=3.5cm, anchor=mid east] at (current page.south
  west) {
\includegraphics[width=0.9\textwidth]{img/vae_code.png}
  };
  \node [xshift=3.00cm, yshift=3.25cm, anchor=mid east] at (current page.south
  west) {
\includegraphics[width=0.20\textwidth]{img/vae_graph.png}
  };
\end{tikzpicture}
\end{frame}

\begin{frame}[plain,t]
\frametitle{Generative Adversarial Network}
\begin{tikzpicture}[remember picture,overlay]
  \node [xshift=12.50cm, yshift=2.8cm, anchor=mid east] at (current page.south
  west) {
\includegraphics[width=0.85\textwidth]{img/gan_code.png}
  };
  \node [xshift=3.20cm, yshift=3.25cm, anchor=mid east] at (current page.south
  west) {
\includegraphics[width=0.20\textwidth]{img/gan_graph.png}
  };
\end{tikzpicture}
\end{frame}

\end{document}
