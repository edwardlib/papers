\documentclass[10pt,
               xcolor={usenames,dvipsnames},
               hyperref={colorlinks,linktoc=all,citecolor=Plum,linkcolor=MidnightBlue,urlcolor=MidnightBlue},noamssymb]{beamer}
\usepackage{etex}
\usepackage[T1]{fontenc}
\usepackage[bitstream-charter]{mathdesign}
\usepackage[scaled=0.92]{PTSans}
\renewcommand*\ttdefault{lmvtt}
\usepackage{exscale}

\usepackage[final,expansion=alltext]{microtype}
\usepackage[english]{babel}
\usepackage{ragged2e}

\usepackage{amsmath}
\usepackage{graphicx}                               % graphics
\usepackage{booktabs, array}                        % nice tables
\usepackage[algoruled]{algorithm2e}                 % pseudocode
\usepackage[numbers,sort]{natbib}                   % better bibliography
\usepackage{cleveref}                               % the best package ever
\usepackage[acronym,smallcaps,nowarn]{glossaries}   % second best package ever

% COLORS
\newcommand{\gray}[1]{\textcolor{black!60}{#1}}
\newcommand{\blu}[1]{\textcolor{jblue}{#1}}

\newacronym{ELBO}{elbo}{evidence lower bound}
\newacronym{KL}{kl}{Kullback-Leibler}

\newacronym{DEF}{def}{deep exponential family}
\newacronym{DLGM}{dlgm}{deep latent Gaussian model}
\newacronym{DRAW}{draw}{Deep Recurrent Attentive Writer}

\newacronym{MF}{mf}{mean-field}
\newacronym{VGP}{vgp}{variational Gaussian process}
\newacronym{BBVI}{bbvi}{black box variational inference}
\newacronym{ADVI}{advi}{automatic differentiation variational inference}
\newacronym{NF}{nf}{normalizing flows}
\newacronym{CVI}{cvi}{copula variational inference}
\newacronym{VAE}{vae}{variational autoencoder}
\newacronym{IWAE}{iwae}{importance weighted autoencoder}
\newacronym{NVIL}{nvil}{neural variational inference}
\newacronym{MIXTURE}{mixture}{}
\newacronym{DSVI}{dsvi}{}

\newacronym{VI}{vi}{variational inference}
\newacronym{EP}{ep}{expectation propagation}

\newacronym{LS}{ls}{Langevin-Stein}
\newacronym{OPVI}{opvi}{operator variational inference}

\DeclareRobustCommand{\mb}[1]{\ensuremath{\boldsymbol{\mathbf{#1}}}}
\DeclareRobustCommand{\mbf}[1]{\ensuremath{\textbf{#1}}}
\DeclareMathOperator*{\argmax}{arg\,max}
\DeclareMathOperator*{\argmin}{arg\,min}


\newcommand{\KL}[2]{\ensuremath{\textrm{KL}\left(#1\;\|\;#2\right)}}

\renewcommand{\mid}{~\vert~}

\newcommand{\mbw}{\mb{w}}
\newcommand{\mbW}{\mb{W}}

\newcommand{\mbx}{\mb{x}}
\newcommand{\mbX}{\mbf{X}}

\newcommand{\mby}{\mb{y}}
\newcommand{\mbY}{\mbf{Y}}

\newcommand{\mbz}{\mb{z}}
\newcommand{\mbZ}{\mb{Z}}

\newcommand{\mbI}{\mbf{I}}
\newcommand{\mbone}{\mbf{1}}

\newcommand{\mbL}{\mbf{L}}

\newcommand{\mbtheta}{\mb{\theta}}
\newcommand{\mbTheta}{\mb{\Theta}}
\newcommand{\mbomega}{\mb{\omega}}
\newcommand{\mbOmega}{\mb{\Omega}}
\newcommand{\mbsigma}{\mb{\sigma}}
\newcommand{\mbSigma}{\mb{\Sigma}}
\newcommand{\mbphi}{\mb{\phi}}
\newcommand{\mbPhi}{\boldsymbol{\Phi}}
% \newcommand{\mblambda}{\mb{\lambda}}

\newcommand{\mbalpha}{\mb{\alpha}}
\newcommand{\mbbeta}{\mb{\beta}}
\newcommand{\mbgamma}{\mb{\gamma}}
\newcommand{\mbeta}{\mb{\eta}}
\newcommand{\mbmu}{\mb{\mu}}
\newcommand{\mbrho}{\mb{\rho}}
\newcommand{\mblambda}{\mb{\lambda}}
\newcommand{\mbzeta}{\mb{\zeta}}

\newcommand\dif{\mathop{}\!\mathrm{d}}
\newcommand{\diag}{\textrm{diag}}
\newcommand{\supp}{\textrm{supp}}

\newcommand{\E}{\mathbb{E}}
\newcommand{\bbH}{\mathbb{H}}
\newcommand{\Var}{\mathbb{V}\textrm{ar}}

\newcommand{\bbN}{\mathbb{N}}
\newcommand{\bbZ}{\mathbb{Z}}
\newcommand{\bbR}{\mathbb{R}}
\newcommand{\bbS}{\mathbb{S}}

\newcommand{\cL}{\mathcal{L}}

\newcommand{\cN}{\mathcal{N}}
\newcommand{\cT}{\mathcal{T}}
\newcommand{\Gam}{\textrm{Gam}}
\newcommand{\InvGam}{\textrm{InvGam}}

\usepackage{tikz}
\usetikzlibrary{bayesnet}

\pgfdeclarelayer{edgelayer}
\pgfdeclarelayer{nodelayer}
\pgfsetlayers{edgelayer,nodelayer,main}

\definecolor{hexcolor0xbfbfbf}{rgb}{0.749,0.749,0.749}

\tikzset{>=latex}
\tikzstyle{none}   = [inner sep=0pt]
\tikzstyle{line}   = [-,
                      thick,
                      shorten <=1pt,
                      shorten >=1pt]
\tikzstyle{arrow}  = [->,
                      thick,
                      shorten <=1pt,
                      shorten >=1pt]
\tikzstyle{ardash} = [dashed,
                      ->,
                      thick,
                      shorten <=1pt,
                      shorten >=1pt]

\tikzstyle{empty}=[
                   circle,
                   opacity=0.0,
                   text opacity=1.0,
                   inner sep=0pt
                  ]

\tikzstyle{box}=[
                 rectangle,
                 fill=White,
                 thick,
                 draw=Black,
                 inner sep=7pt
                ]

\tikzstyle{filled}=[
                    circle,
                    thick,
                    fill=hexcolor0xbfbfbf,
                    draw=Black
                   ]

\tikzstyle{hollow}=[
                    circle,
                    thick,
                    fill=White,
                    draw=Black
                   ]

\tikzstyle{param}=[
                   rectangle,
                   fill=Black,
                   draw=Black,
                   inner sep=0pt,
                   minimum width=4pt,
                   minimum height=4pt
                  ]

\tikzstyle{paramhollow}=[
                         rectangle,
                         thick,
                         fill=White,
                         draw=Black,
                         inner sep=0pt,
                         minimum
                         width=4pt,
                         minimum height=4pt
                        ]

\usepackage{pgfplots}                               % PGFPLOTS baby!
\pgfplotsset{compat=newest}
\pgfplotsset{plot coordinates/math parser=false}
% \usepgfplotslibrary{statistics}


\usepackage{subfigure}
\definecolor{light}{RGB}{199, 153, 199}
\definecolor{dark}{RGB}{143, 39, 143}
\definecolor{gray80}{gray}{0.8}

\usepackage{natbib}
\setbeamertemplate{navigation symbols}{}
\setbeamertemplate{itemize items}[circle]
\setbeamercolor{itemize item}{fg=black!67}
\setbeamercolor*{enumerate item}{fg=black!67}
\setbeamercolor*{enumerate subitem}{fg=black!67}
\setbeamercolor*{enumerate subsubitem}{fg=black!67}

\definecolor{charcoal}{HTML}{222222}
\definecolor{snow}{HTML}{F9F9F9}

\setbeamercolor{background canvas}{bg=white}
\setbeamercolor{normal text}{fg=charcoal}
\setbeamercolor{structure}{fg=charcoal}

\newenvironment{changemargin}[1]{
  \begin{list}{}{
    \setlength{\topsep}{0pt}
    \setlength{\leftmargin}{#1}
    \setlength{\rightmargin}{#1}
    \setlength{\listparindent}{\parindent}
    \setlength{\itemindent}{\parindent}
    \setlength{\parsep}{\parskip}
  }
  \item[]}{\end{list}}

\title{}
\begin{document}

% it's clear that there needs to be two types of talks for each
% audience
% + engineers
% + researchers
% todo
% + visuals/criticism after code examples, e.g., what a fit looks like
%   + accuracy results, plot of elbo after running
% + explain how to do finer control of inference
% + batch training

\begin{frame}[plain,t]
\begin{tikzpicture}[remember picture,overlay]
  \node [xshift=0.50cm, yshift=-3.00cm, anchor=north west] at (current page.north west) {
    \begin{tabular}{l}
    {\Large\bf Edward: A library for probabilistic modeling,}\\[1ex]
    {\Large\bf inference, and criticism}\\[2ex]
    {\large }\\[4ex]
    Dustin Tran\\
    Columbia University\\[4ex]
    \end{tabular}
  };
  \node [xshift=-1.50cm, yshift=3.00cm, anchor=mid east] at (current page.south
  east) {
\includegraphics[width=0.25\textwidth]{img/edward.png}
  };
\end{tikzpicture}
\end{frame}

\begin{frame}[plain]
\footnotesize
\begin{tikzpicture}[remember picture,overlay]
  \node [xshift=-0.3cm, yshift=-1.0cm, anchor=north west] at (current page.north west) {
\begin{tabular}{ccccc}
\includegraphics[width=22mm]{img/alp.png} &
\includegraphics[width=22mm]{img/adji.jpg} &
\includegraphics[width=22mm]{img/maja.png} &
\includegraphics[width=22mm]{img/dawen.jpg} &
\includegraphics[width=22mm]{img/blei.jpg}\\
Alp Kucukelbir & Adji Dieng & Maja Rudolph & Dawen Liang & David Blei\\
\end{tabular}
  };
\end{tikzpicture}

\vspace{30ex}
\begin{center}
\vspace*{5mm}

\begin{tabular}{cccc}
\includegraphics[width=22mm]{img/matt.jpg} &
\includegraphics[width=22mm]{img/kevin.jpg} &
\includegraphics[width=22mm]{img/eugene.jpg} &
\includegraphics[width=22mm]{img/rif.png}\\
Matt Hoffman & Kevin Murphy & Eugene Brevdo & Rif Saurous\\
\end{tabular}
\end{center}
\end{frame}

\begin{frame}[plain,t]
\vspace{2ex}
\begin{center}
\includegraphics[width=0.67\textwidth]{img/nytimes.png}
\\[2ex]
Topics found in 1.8M articles from the New York Times
\end{center}
\begin{tikzpicture}[remember picture,overlay]
  \node [xshift=-5.0cm, yshift=0.4cm, anchor=south west] at (current
  page.south east) {
\gray{\small [Hoffman, Blei, Wang, Paisley 2013]}
  };
\end{tikzpicture}
\end{frame}

\begin{frame}[plain,t]
\begin{center}
\includegraphics[width=0.9\textwidth]{img/genetics.png}
\\[2ex]
Population analysis of 2 billion genetic measurements
\end{center}
\begin{tikzpicture}[remember picture,overlay]
  \node [xshift=-2.5cm, yshift=0.4cm, anchor=south west] at (current
  page.south east) {
\gray{\small [Gopalan+ 2017]}
  };
\end{tikzpicture}
\end{frame}

\begin{frame}[plain,t]
\vspace{15ex}
\begin{center}
\includegraphics[width=1.0\textwidth]{img/control.png}
\\[2ex]
Understanding scenes, concepts, and control
\end{center}
\begin{tikzpicture}[remember picture,overlay]
  \node [xshift=-4.0cm, yshift=0.4cm, anchor=south west] at (current
  page.south east) {
\gray{\small [Eslami+ 2016; Lake+ 2015]}
  };
\end{tikzpicture}
\end{frame}

\begin{frame}[plain,t]
\vspace{2ex}
\begin{center}
\includegraphics[width=0.9\textwidth]{img/taxis.png}
\\[2ex]
Exploratory analysis of 1.7M taxi trajectories, in Stan
\end{center}
\begin{tikzpicture}[remember picture,overlay]
  \node [xshift=-3.0cm, yshift=0.4cm, anchor=south west] at (current
  page.south east) {
\gray{\small [Kucukelbir+ 2017]}
  };
\end{tikzpicture}
\end{frame}

\begin{frame}[plain,t]
\vspace{15ex}
\begin{center}
\includegraphics[width=1.0\textwidth]{img/vision.png}
\\[2ex]
Compression and content generation
\end{center}
\begin{tikzpicture}[remember picture,overlay]
  \node [xshift=-5.0cm, yshift=0.4cm, anchor=south west] at (current
  page.south east) {
\gray{\small [Van der Oord+ 2016; Gregor+ 2016]}
  };
\end{tikzpicture}
\end{frame}

\begin{frame}
\frametitle{George E.P. Box (1919 - 2013)}
\begin{columns}
\begin{column}{0.5\textwidth}
    \begin{center}
     \includegraphics[width=\columnwidth]{img/box.jpg}
     \end{center}
\end{column}
\begin{column}{0.5\textwidth}
An iterative process for science:
\\[1ex]
\begin{enumerate}
\item Build a model of the science
\\[1ex]
\item Infer the model given data
\\[1ex]
\item Criticize the model given data
\end{enumerate}
\end{column}
\end{columns}
\begin{tikzpicture}[remember picture,overlay]
  \node [xshift=-8.0cm, yshift=0.4cm, anchor=south west] at (current
  page.south east) {
\gray{\small [Box \& Hunter 1962, 1965; Box \& Hill 1967; Box 1976, 1980]}
  };
\end{tikzpicture}
\end{frame}

\begin{frame}
\frametitle{Box's Loop}
\begin{tikzpicture}[remember picture,overlay]
  \node [xshift=-1cm, yshift=-2.00cm, anchor=north west] at (current page.north west) {
\includegraphics[width=1.4\textwidth]{img/model_infer_criticize.png}
  };
  \node [xshift=3.4cm, yshift=-8.0cm, anchor=north west] at (current page.north west) {
Edward is a library designed around this loop.
  };
  \node [xshift=0cm, yshift=-5.50cm, anchor=north west] at (current page.north west) {
  };
  \node [xshift=-4.0cm, yshift=0.4cm, anchor=south west] at (current
  page.south east) {
\gray{\small [Box 1976, 1980; Blei 2014]}
  };
\end{tikzpicture}
\end{frame}

\begin{frame}
\vspace{3ex}
\textbf{Edward} is a probabilistic programming language
built on TensorFlow.

\emph{Modeling}
\begin{itemize}
\item
Composable Turing-complete language of random variables.
\item
Many data types, tensor vectorization, broadcasting, 3rd party support.
\item
Examples:
Graphical models, neural networks, probabilistic programs.
\end{itemize}

\emph{Inference}
\begin{itemize}
\item
Composable language for hybrids, message passing, data subsampling.
\item
Infrastructure to develop your own algorithms.
\item
Examples:
Black box VI, Hamiltonian MC, stochastic
gradient MCMC.
\end{itemize}

\emph{Criticism}
\begin{itemize}
\item
Examples: Scoring rules, hypothesis tests, predictive checks.
\end{itemize}

\vspace{1ex}
Features include autodiff, multi-GPUs, distributed, XLA, quantization.

\begin{tikzpicture}[remember picture,overlay]
  \node [xshift=-3.0cm, yshift=0.4cm, anchor=south west] at (current
  page.south east) {
\gray{\small [Tran+ 2016, 2017]}
  };
\end{tikzpicture}
\end{frame}

\begin{frame}
\begin{center}
\vspace{-3.5ex}
\includegraphics[width=1.1\textwidth]{img/github.png}
\\[-7ex]
\includegraphics[width=1.0\textwidth]{img/forum.png}
\\[-3ex]
\includegraphics[width=1.0\textwidth]{img/gitter.png}
\\[2ex]
\end{center}
\text{
We have an active community of several hundred users \& many
contributors.
}
\end{frame}

\begin{frame}
\frametitle{Who is Using Edward?}
{\large Users}
\begin{enumerate}
\item
Machine learning enthusiasts, data scientists, business analysts \\
(\emph{ex. hierarchical GLMs, mixture models, MAP, MCMC, ...})
\item
Probabilistic graphical modeling community \\
(\emph{ex. latent Dirichlet allocation, variational inference, Gibbs})
\item
Bayesian deep learning community \\
(\emph{ex. deep generative models, Bayesian NNs, black box inference})
\end{enumerate}

{\large Developers}
\begin{enumerate}
\item
David Blei's group
\item
Google Brain
(\emph{in conception/design})
\item
Matt Hoffman (\emph{conjugacy}),
Emily Fox's group
(\emph{time series + SGMCMC}),
Justin Bayer (\emph{stochastic RNNs}),
John Pearson (\emph{neuroscience}),
a few Master's/Ph.D. students.
\item
Collaboration continues to evolve. Contact us! (+visit the Forum)
\end{enumerate}
\end{frame}

\begin{frame}
\frametitle{How do we use Edward?}
\vspace{10ex}
\begin{center}
\gray{\Large [Demo]} \\[3ex]
{\large \url{edwardlib.org/getting-started}}
\end{center}
\end{frame}

\begin{frame}
\frametitle{Model}
A random variable $\mbx$ is an object parameterized by tensors
(multi-dimensional arrays) $\theta^*$.

\vspace{-1.0ex}
\includegraphics[height=0.20\textwidth]{img/random_variables.png}

It is equipped with methods such as \texttt{log\_prob()} and \texttt{sample()}.

Each random variable is associated to a tensor $\mbx^*$, $\mbx^*\sim p(\mbx\g\theta^*)$.

Mutable states let random variables condition on values that
change, e.g., discriminative models $p(\mby\g\mbx)$ and model
parameters $p(\mbx; \theta)$.
\end{frame}

\begin{frame}
\frametitle{Example: Beta-Bernoulli}
Consider a Beta-Bernoulli model,
\begin{align*}
p(\mbx, \theta) =
\operatorname{Beta}(\theta\g 1, 1)
\prod_{n=1}^{50} \operatorname{Bernoulli}(x_n\g \theta),
\end{align*}
where $\theta$
is a probability shared across 50 data points $\mbx\in\{0,1\}^{50}$.
\begin{center}
\vspace{-2ex}
\includegraphics[height=0.175\textwidth]{img/beta-bernoulli.png}
\end{center}
Fetching $\mbx$ from the graph generates a binary vector of $50$ elements.

All computation is represented on the graph, enabling us to leverage model structure during inference.
\end{frame}

\begin{frame}
\frametitle{Example: Gaussian Matrix Factorization}
\begin{center}
\includegraphics[height=0.3\textwidth]{img/gaussian_mf_graph.png}
\\
\includegraphics[height=0.2\textwidth]{img/gaussian_mf_program.png}
\end{center}
\begin{tikzpicture}[remember picture,overlay]
  \node [xshift=-7.0cm, yshift=0.4cm, anchor=south west] at (current
  page.south east) {
\gray{\small [Bishop \& Tipping 1997; Salakhutdinov \& Mnih 2007]}
  };
\end{tikzpicture}
\end{frame}

\begin{frame}
\frametitle{Example: Variational Auto-Encoder for Binarized MNIST}
\begin{tikzpicture}[remember picture,overlay]
  \node [xshift=0.4cm, yshift=-0.8cm, anchor=north west] at (current page.north west) {
   \includegraphics[width=0.5\columnwidth]{img/vae_graphical_model_01.png}
  };
  \node [xshift=-5.75cm, yshift=0.4cm, anchor=south west] at (current
  page.south east) {
\gray{\small [Kingma \& Welling 2014; Rezende+ 2014]}
  };
\end{tikzpicture}
\end{frame}

\begin{frame}
\frametitle{Example: Variational Auto-Encoder for Binarized MNIST}
\begin{tikzpicture}[remember picture,overlay]
  \node [xshift=0.5cm, yshift=-1.0cm, anchor=north west] at (current page.north west) {
   \includegraphics[width=0.18\columnwidth]{img/vae_graphical_model.png}
  };
  \node [xshift=2.6cm, yshift=-1.7cm, anchor=north west] at (current page.north west) {
\includegraphics[height=0.14\textheight]{img/vae_decoder.png}
  };
  \node [xshift=2.6cm, yshift=-5.8cm, anchor=north west] at (current page.north west) {
\includegraphics[height=0.18\textheight]{img/vae_encoder.png}
  };
  \node [xshift=-5.75cm, yshift=0.4cm, anchor=south west] at (current
  page.south east) {
\gray{\small [Kingma \& Welling 2014; Rezende+ 2014]}
  };
\end{tikzpicture}
\end{frame}

\begin{frame}
\frametitle{Example: Variational Auto-Encoder for Binarized MNIST}
\vspace{10ex}
\begin{center}
\gray{\Large [Demo]} \\[3ex]
{\large \url{github.com/blei-lab/edward/blob/master/examples/vae.py}}
\end{center}
\end{frame}

\begin{frame}
\frametitle{Inference}
Given
\begin{itemize}
\item Data $\mbx_{\text{train}}$.
\item
Model $p(\mbx, \mbz, \mbbeta)$ of
observed variables $\mbx$ and latent variables $\mbz, \mbbeta$.
\end{itemize}
Goal
\begin{itemize}
\item
Calculate posterior distribution
\begin{equation*}
p(\mbz, \mbbeta\mid\mbx_{\text{train}}) =
\frac{p(\mbx_{\text{train}}, \mbz, \mbbeta)}{\int
p(\mbx_{\text{train}}, \mbz, \mbbeta) \d\mbz\d\mbbeta}.
\end{equation*}
\end{itemize}
\vspace{2ex}
This is the key problem in Bayesian inference.
\begin{tikzpicture}[remember picture,overlay]
  \node [xshift=-4.5cm, yshift=0.4cm, anchor=south west] at (current
  page.south east) {
\small \url{edwardlib.org/tutorials}
  };
\end{tikzpicture}
\end{frame}

\begin{frame}
\frametitle{Inference}
% Let \texttt{z} and \texttt{beta} be latent variables in the model,
% where we observe the random variable \texttt{x} with
% data \texttt{x\_train}.
% Let \texttt{qz} and \texttt{qbeta} be random variables defined to
% approximate the posterior.

In Edward, all inference has the same structure.

\begin{center}
\vspace{-2.0ex}
\includegraphics[height=0.05\textheight]{img/inference.png}
\end{center}

\texttt{Inference} has two inputs: \\
\begin{enumerate}
\vspace{-3ex}
\item
latent variables $\mbz,\beta$, binding model variables to approximate factors;
\item
observed variables $\mbx$, binding model variables to data.
\end{enumerate}

\texttt{Inference} has class methods:
\begin{itemize}
\item
\texttt{run()} runs the algorithm from initialization to convergence;
\item
\texttt{initialize()}, \texttt{update()}, \texttt{print\_progress()}, etc.
provides finer control.
\end{itemize}
\end{frame}

\begin{frame}
\frametitle{Inference}
Variational inference. It uses a variational model.
\begin{center}
\vspace{-2.0ex}
\includegraphics[height=0.18\textheight]{img/inference_variational.png}
\end{center}
Monte Carlo. It uses an Empirical approximation.
\begin{center}
\includegraphics[height=0.17\textheight]{img/inference_monte.png}
\end{center}

Conjugacy \& exact inference. It uses symbolic algebra on the graph.
\end{frame}

\begin{frame}[t]
\frametitle{Example: Variational Auto-Encoder for Binarized MNIST}
\vspace{5ex}
\begin{center}
\includegraphics[width=0.975\textwidth]{img/vae_example.png}
\end{center}
\end{frame}

\begin{frame}[t]
\frametitle{Example: Variational Auto-Encoder for Binarized MNIST}
\vspace{2.5ex}
\begin{center}
\includegraphics[width=1.05\textwidth]{img/vae_example_hmc.png}
\end{center}
\end{frame}

\begin{frame}
\frametitle{Non-Bayesian Methods: GANs}
GANs posit a generative process,
\begin{align*}
\mbepsilon &\sim \text{Normal}(0, 1) \\
\mbx &= G(\mbepsilon;\theta)
\end{align*}
for some generative network $G$.

Training uses a discriminative network $D$ via the
optimization problem
\begin{equation*}
\min_\theta \max_\phi
\mathbb{E}_{p^*(\mbx)} [ \log D(\mbx; \phi) ]
+ \mathbb{E}_{p(\mbx; \theta)} [ \log (1 - D(\mbx; \phi)) ]
\end{equation*}
The generator tries to generate samples indistinguishable from true data.

The discriminator tries to discriminate samples from the generator and
samples from the true data.

% We alternate gradient updates over the discriminator and generator's
% parameters.

\begin{tikzpicture}[remember picture,overlay]
  \node [xshift=-3.0cm, yshift=0.4cm, anchor=south west] at (current
  page.south east) {
\gray{\small [Goodfellow+ 2014]}
  };
\end{tikzpicture}
\end{frame}

\begin{frame}
\frametitle{Non-Bayesian Methods: GANs}
\begin{center}
\vspace{-2ex}
\includegraphics[width=1.0\textwidth]{img/gan_example.png}
\end{center}
\begin{tikzpicture}[remember picture,overlay]
  \node [xshift=-3.0cm, yshift=0.4cm, anchor=south west] at (current
  page.south east) {
\gray{\small [Goodfellow+ 2014]}
  };
\end{tikzpicture}
\end{frame}

\begin{frame}
\frametitle{Non-Bayesian Methods: GANs}
\begin{center}
\vspace{-2ex}
\includegraphics[width=1.1\textwidth]{img/ganw_example.png}
\end{center}
\begin{tikzpicture}[remember picture,overlay]
  \node [xshift=-4.5cm, yshift=0.4cm, anchor=south west] at (current
  page.south east) {
\gray{\small [Arjovsky+ 2017; Gulrajani+ 2017]}
  };
\end{tikzpicture}
\end{frame}

\begin{frame}
\frametitle{Experiment: GPU-accelerated Hamiltonian Monte Carlo}
\begin{center}
\includegraphics[width=0.85\textwidth]{img/experiments_hmc.png}
\end{center}
\vspace{-1ex}
Run HMC for 100 iterations and fixed hyperparameters.

Bayesian logistic regression for Covertype ($581012$ data points, $54$
features).

12-core Intel i7-5930K CPU at 3.50GHz and NVIDIA Titan X (Maxwell) GPU.
Single precision.

\vspace{4ex}
\textbf{Edward is orders of magnitude faster than existing software
for large data.}
\begin{tikzpicture}[remember picture,overlay]
  \node [xshift=-2cm, yshift=0.4cm, anchor=south west] at (current
  page.south east) {
\gray{\small [Tran+ 2017]}
  };
\end{tikzpicture}
\end{frame}

\begin{frame}
\frametitle{Experiment: Recent Methods in Variational Inference}
\begin{center}
\includegraphics[height=0.4\textwidth]{img/experiments_vi.png}
\end{center}
\vspace{2ex}
\textbf{Edward enables fast experimentation with state-of-the-art methods.}
\begin{tikzpicture}[remember picture,overlay]
  \node [xshift=-2cm, yshift=0.4cm, anchor=south west] at (current
  page.south east) {
\gray{\small [Tran+ 2017]}
  };
\end{tikzpicture}
\end{frame}

\begin{frame}[c]
\frametitle{Summary}
\begin{enumerate}
\item
Edward is a library designed for fast experimentation.

It emphasizes fine tuning of inference alongside model development.
\item
Edward is integrated into TensorFlow.

It features significant speedups over existing systems, bridging
research design and deployment.
\end{enumerate}
\end{frame}

\begin{frame}
\frametitle{Current directions}

We are robustifying Edward.
\begin{enumerate}
\item Bug fixes, improved vectorization/types, errors/warnings.
\item More built-in algorithms and model / inference tutorials.
\item More automation (ex. transformations, visualization).
\end{enumerate}
\vspace{3ex}

We are applying Edward for modeling, inference, and scientific research.
\begin{enumerate}
\vspace{-1.5ex}
\item
Large scale applications in genome wide association studies.
\item
Bayesian inference for GANs (simulator-based models).
\item
Checking approximate inferences.
\end{enumerate}
\end{frame}

\begin{frame}
\frametitle{References}
\begin{center}
\includegraphics[width=0.3\textwidth]{img/edward.png}
\\
\large \url{edwardlib.org}
\end{center}
\vspace{2ex}

\begin{itemize}
\item
\textbf{D.~Tran}, A.~Kucukelbir, A.~Dieng, M.~Rudolph, D.~Liang, and
D.M.~Blei.
Edward: A library for probabilistic modeling, inference, and criticism.
\gray{arXiv preprint arXiv:1610.09787, 2016.}
\item
\textbf{D.~Tran}, M.D.~Hoffman, R.A.~Saurous, E.~Brevdo, K.~Murphy, and
D.M.~Blei.
Deep probabilistic programming. \\
\gray{International Conference on Learning Representations, 2017.}
\end{itemize}
\end{frame}

% \begin{frame}
% \end{frame}

% \begin{frame}
% \frametitle{Example: Latent Dirichlet allocation}
% \begin{center}
% \includegraphics[height=0.3\textwidth]{img/lda_graph.png}
% \\[2ex]
% \includegraphics[height=0.4\textwidth]{img/lda_program.png}
% \end{center}
% \begin{tikzpicture}[remember picture,overlay]
%   \node [xshift=-3.25cm, yshift=0.4cm, anchor=south west] at (current
%   page.south east) {
% \gray{\small [Blei, Ng, Jordan 2003]}
%   };
% \end{tikzpicture}
% \end{frame}

% \begin{frame}
% \frametitle{Example: Bayesian recurrent neural network}
% \begin{center}
% \includegraphics[height=0.32\textwidth]{img/bayesian_rnn_graph.png}
% \includegraphics[height=0.32\textwidth]{img/bayesian_rnn_program.png}
% \end{center}
% \end{frame}

% \begin{frame}
% \frametitle{Example: Bayesian neural network for classification}
% \begin{center}
% \includegraphics[height=0.3\textwidth]{img/bayesian_nn_graph.png}
% \\[2.5ex]
% \includegraphics[height=0.19\textwidth]{img/bayesian_nn_program.png}
% \end{center}
% \end{frame}

% \begin{frame}
% \frametitle{Inference: Composing Inference}
% Core to Edward's design is that inference can be written as a collection of separate inference programs.

% For example, here is variational EM.

% \begin{center}
% \includegraphics[height=0.275\textheight]{img/composing.png}
% \end{center}

% We can also write message passing algorithms, which work over a collection of local inference problems. This includes expectation propagation.
% \begin{tikzpicture}[remember picture,overlay]
%   \node [xshift=-5.0cm, yshift=0.4cm, anchor=south west] at (current
%   page.south east) {
% \gray{\small [Minka+ 2009; Mansinghka+ 2014]}
%   };
% \end{tikzpicture}
% \end{frame}

% \begin{frame}
% \frametitle{Inference: Data Subsampling}
% Stochastic optimization scales inference to massive data
% \gray{\small [Welling \& Teh 2011; Hoffman+ 2013]}.

% To support such algorithms, we represent only a subgraph of the full model.

% \begin{center}
% \vspace{-3.0ex}
% \includegraphics[height=0.30\textheight]{img/hierarchical_model_subgraph.png}
% \end{center}
% We then pass in a scale argument during initialization,

% \begin{center}
% \vspace{-1.0ex}
% \includegraphics[height=0.11\textheight]{img/inference_subsampling_1.png}
% \end{center}
% \end{frame}

\end{document}
