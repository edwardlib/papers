\documentclass[10pt,
               xcolor={usenames,dvipsnames},
               hyperref={colorlinks,linktoc=all,citecolor=Plum,linkcolor=MidnightBlue,urlcolor=MidnightBlue},noamssymb]{beamer}
\usepackage{etex}
\usepackage[T1]{fontenc}
\usepackage[bitstream-charter]{mathdesign}
\usepackage[scaled=0.92]{PTSans}
\renewcommand*\ttdefault{lmvtt}
\usepackage{exscale}

\usepackage[final,expansion=alltext]{microtype}
\usepackage[english]{babel}
\usepackage{ragged2e}

\usepackage{amsmath}
\usepackage{graphicx}                               % graphics
\usepackage{booktabs, array}                        % nice tables
\usepackage[algoruled]{algorithm2e}                 % pseudocode
\usepackage[numbers,sort]{natbib}                   % better bibliography
\usepackage{cleveref}                               % the best package ever
\usepackage[acronym,smallcaps,nowarn]{glossaries}   % second best package ever

% COLORS
\newcommand{\gray}[1]{\textcolor{black!60}{#1}}
\newcommand{\blu}[1]{\textcolor{jblue}{#1}}

\newacronym{ELBO}{elbo}{evidence lower bound}
\newacronym{KL}{kl}{Kullback-Leibler}

\newacronym{DEF}{def}{deep exponential family}
\newacronym{DLGM}{dlgm}{deep latent Gaussian model}
\newacronym{DRAW}{draw}{Deep Recurrent Attentive Writer}

\newacronym{MF}{mf}{mean-field}
\newacronym{VGP}{vgp}{variational Gaussian process}
\newacronym{BBVI}{bbvi}{black box variational inference}
\newacronym{ADVI}{advi}{automatic differentiation variational inference}
\newacronym{NF}{nf}{normalizing flows}
\newacronym{CVI}{cvi}{copula variational inference}
\newacronym{VAE}{vae}{variational autoencoder}
\newacronym{IWAE}{iwae}{importance weighted autoencoder}
\newacronym{NVIL}{nvil}{neural variational inference}
\newacronym{MIXTURE}{mixture}{}
\newacronym{DSVI}{dsvi}{}

\newacronym{VI}{vi}{variational inference}
\newacronym{EP}{ep}{expectation propagation}

\newacronym{LS}{ls}{Langevin-Stein}
\newacronym{OPVI}{opvi}{operator variational inference}

\DeclareRobustCommand{\mb}[1]{\ensuremath{\boldsymbol{\mathbf{#1}}}}
\DeclareRobustCommand{\mbf}[1]{\ensuremath{\textbf{#1}}}
\DeclareMathOperator*{\argmax}{arg\,max}
\DeclareMathOperator*{\argmin}{arg\,min}


\newcommand{\KL}[2]{\ensuremath{\textrm{KL}\left(#1\;\|\;#2\right)}}

\renewcommand{\mid}{~\vert~}

\newcommand{\mbw}{\mb{w}}
\newcommand{\mbW}{\mb{W}}

\newcommand{\mbx}{\mb{x}}
\newcommand{\mbX}{\mbf{X}}

\newcommand{\mby}{\mb{y}}
\newcommand{\mbY}{\mbf{Y}}

\newcommand{\mbz}{\mb{z}}
\newcommand{\mbZ}{\mb{Z}}

\newcommand{\mbI}{\mbf{I}}
\newcommand{\mbone}{\mbf{1}}

\newcommand{\mbL}{\mbf{L}}

\newcommand{\mbtheta}{\mb{\theta}}
\newcommand{\mbTheta}{\mb{\Theta}}
\newcommand{\mbomega}{\mb{\omega}}
\newcommand{\mbOmega}{\mb{\Omega}}
\newcommand{\mbsigma}{\mb{\sigma}}
\newcommand{\mbSigma}{\mb{\Sigma}}
\newcommand{\mbphi}{\mb{\phi}}
\newcommand{\mbPhi}{\boldsymbol{\Phi}}
% \newcommand{\mblambda}{\mb{\lambda}}

\newcommand{\mbalpha}{\mb{\alpha}}
\newcommand{\mbbeta}{\mb{\beta}}
\newcommand{\mbgamma}{\mb{\gamma}}
\newcommand{\mbeta}{\mb{\eta}}
\newcommand{\mbmu}{\mb{\mu}}
\newcommand{\mbrho}{\mb{\rho}}
\newcommand{\mblambda}{\mb{\lambda}}
\newcommand{\mbzeta}{\mb{\zeta}}

\newcommand\dif{\mathop{}\!\mathrm{d}}
\newcommand{\diag}{\textrm{diag}}
\newcommand{\supp}{\textrm{supp}}

\newcommand{\E}{\mathbb{E}}
\newcommand{\bbH}{\mathbb{H}}
\newcommand{\Var}{\mathbb{V}\textrm{ar}}

\newcommand{\bbN}{\mathbb{N}}
\newcommand{\bbZ}{\mathbb{Z}}
\newcommand{\bbR}{\mathbb{R}}
\newcommand{\bbS}{\mathbb{S}}

\newcommand{\cL}{\mathcal{L}}

\newcommand{\cN}{\mathcal{N}}
\newcommand{\cT}{\mathcal{T}}
\newcommand{\Gam}{\textrm{Gam}}
\newcommand{\InvGam}{\textrm{InvGam}}

\usepackage{tikz}
\usetikzlibrary{bayesnet}

\pgfdeclarelayer{edgelayer}
\pgfdeclarelayer{nodelayer}
\pgfsetlayers{edgelayer,nodelayer,main}

\definecolor{hexcolor0xbfbfbf}{rgb}{0.749,0.749,0.749}

\tikzset{>=latex}
\tikzstyle{none}   = [inner sep=0pt]
\tikzstyle{line}   = [-,
                      thick,
                      shorten <=1pt,
                      shorten >=1pt]
\tikzstyle{arrow}  = [->,
                      thick,
                      shorten <=1pt,
                      shorten >=1pt]
\tikzstyle{ardash} = [dashed,
                      ->,
                      thick,
                      shorten <=1pt,
                      shorten >=1pt]

\tikzstyle{empty}=[
                   circle,
                   opacity=0.0,
                   text opacity=1.0,
                   inner sep=0pt
                  ]

\tikzstyle{box}=[
                 rectangle,
                 fill=White,
                 thick,
                 draw=Black,
                 inner sep=7pt
                ]

\tikzstyle{filled}=[
                    circle,
                    thick,
                    fill=hexcolor0xbfbfbf,
                    draw=Black
                   ]

\tikzstyle{hollow}=[
                    circle,
                    thick,
                    fill=White,
                    draw=Black
                   ]

\tikzstyle{param}=[
                   rectangle,
                   fill=Black,
                   draw=Black,
                   inner sep=0pt,
                   minimum width=4pt,
                   minimum height=4pt
                  ]

\tikzstyle{paramhollow}=[
                         rectangle,
                         thick,
                         fill=White,
                         draw=Black,
                         inner sep=0pt,
                         minimum
                         width=4pt,
                         minimum height=4pt
                        ]

\usepackage{pgfplots}                               % PGFPLOTS baby!
\pgfplotsset{compat=newest}
\pgfplotsset{plot coordinates/math parser=false}
% \usepgfplotslibrary{statistics}


\usepackage{subfigure}
\definecolor{light}{RGB}{199, 153, 199}
\definecolor{dark}{RGB}{143, 39, 143}
\definecolor{gray80}{gray}{0.8}

\usepackage{natbib}
\setbeamertemplate{navigation symbols}{}
\setbeamertemplate{itemize items}[circle]
\setbeamercolor{itemize item}{fg=black!67}
\setbeamercolor*{enumerate item}{fg=black!67}
\setbeamercolor*{enumerate subitem}{fg=black!67}
\setbeamercolor*{enumerate subsubitem}{fg=black!67}

\definecolor{charcoal}{HTML}{222222}
\definecolor{snow}{HTML}{F9F9F9}

\setbeamercolor{background canvas}{bg=white}
\setbeamercolor{normal text}{fg=charcoal}
\setbeamercolor{structure}{fg=charcoal}

\newenvironment{changemargin}[1]{
  \begin{list}{}{
    \setlength{\topsep}{0pt}
    \setlength{\leftmargin}{#1}
    \setlength{\rightmargin}{#1}
    \setlength{\listparindent}{\parindent}
    \setlength{\itemindent}{\parindent}
    \setlength{\parsep}{\parskip}
  }
  \item[]}{\end{list}}

\title{}
\begin{document}

\begin{frame}[plain,t]
\begin{tikzpicture}[remember picture,overlay]
  \node [xshift=0.50cm, yshift=-3.00cm, anchor=north west] at (current page.north west) {
    \begin{tabular}{l}
    {\Large\bf An Overview of Edward:}\\[1ex]
    {\Large\bf A Probabilistic Programming System}\\[2ex]
    {\large }\\[4ex]
    Dustin Tran\\
    Columbia University \\[4ex]
    \end{tabular}
  };
  \node [xshift=-1.50cm, yshift=3.00cm, anchor=mid east] at (current page.south
  east) {
\includegraphics[width=0.25\textwidth]{img/edward.png}
  };
  \node [xshift=-0.25cm, yshift=0.5cm, anchor=mid east] at (current page.south
  east) {
    \includegraphics[width=0.22\textwidth]{img/columbia.pdf}
  };
\end{tikzpicture}
\end{frame}

\begin{frame}[plain]
\footnotesize
\begin{center}
\begin{tabular}{ccc}
\includegraphics[width=22mm]{img/alp.png} &
\includegraphics[width=22mm]{img/adji.jpg} &
\includegraphics[width=22mm]{img/dawen.jpg} \\
Alp Kucukelbir & Adji Dieng & Dawen Liang \\
\end{tabular}

\vspace{-2ex}

\begin{tabular}{cccc}
\includegraphics[width=22mm]{img/eugene.jpg} &
\includegraphics[width=22mm]{img/maja.png} &
\includegraphics[width=22mm]{img/matt.jpg} &
\includegraphics[width=22mm]{img/ranganath.jpg} \\
Eugene Brevdo & Maja Rudolph & Matt Hoffman & Rajesh Ranganath \\
\end{tabular}

\vspace{-2ex}

\begin{tabular}{cccc}
\includegraphics[width=20mm]{img/gelman.png} &
\includegraphics[width=20mm]{img/blei.jpg} &
\includegraphics[width=20mm]{img/kevin.jpg} &
\includegraphics[width=20mm]{img/rif.png}\\
Andrew Gelman & David Blei & Kevin Murphy & Rif Saurous\\
\end{tabular}
\end{center}
\end{frame}

% \begin{frame}[plain,t]
% \vspace{2ex}
% \begin{center}
% \includegraphics[width=0.67\textwidth]{img/nytimes.png}
% \\[2ex]
% Topics found in 1.8M articles from the New York Times
% \end{center}
% \begin{tikzpicture}[remember picture,overlay]
%   \node [xshift=-5.0cm, yshift=0.4cm, anchor=south west] at (current
%   page.south east) {
% \gray{\small [Hoffman, Blei, Wang, Paisley 2013]}
%   };
% \end{tikzpicture}
% \end{frame}

\begin{frame}[plain,t]
\vspace{2ex}
\begin{center}
\includegraphics[width=0.9\textwidth]{img/taxis.png}
\\[2ex]
Exploratory analysis of 1.7M taxi trajectories, in Stan
\end{center}
\begin{tikzpicture}[remember picture,overlay]
  \node [xshift=-3.0cm, yshift=0.4cm, anchor=south west] at (current
  page.south east) {
\gray{\small [Kucukelbir+ 2017]}
  };
\end{tikzpicture}
\end{frame}

\begin{frame}[plain,t]
\begin{tikzpicture}[remember picture,overlay]
  \node [xshift=0.25cm, yshift=-1.5cm, anchor=north west] at (current
  page.north west) {
\includegraphics[width=0.6\textwidth]{img/lotka_volterra_plot.png}
  };
  \node [xshift=6.75cm, yshift=-2.0cm, anchor=north west] at (current
  page.north west) {
\includegraphics[width=0.5\textwidth]{img/lotka_volterra_inference.png}
  };
  \node [xshift=2.5cm, yshift=1.0cm, anchor=south west] at (current
  page.south west) {
Simulators of 100K time series in ecology, in Edward
  };
  \node [xshift=-2.25cm, yshift=0.4cm, anchor=south west] at (current
  page.south east) {
\gray{\small [Tran+ 2017]}
  };
\end{tikzpicture}
\end{frame}

\begin{frame}[plain,t]
\vspace{2.0cm}
\begin{center}
\includegraphics[width=0.37\textwidth]{img/dim_graphical_model.png}
\hfill
\includegraphics[width=0.45\textwidth]{img/dim_samples.png}
\\[1.5cm]
Generation \& compression of 10M colored 32x32 images, in Edward
\end{center}
\begin{tikzpicture}[remember picture,overlay]
  \node [xshift=-5.75cm, yshift=0.4cm, anchor=south west] at (current
  page.south east) {
\gray{\small [Tran+ 2017; fig from Van der Oord+ 2016]}
  };
\end{tikzpicture}
\end{frame}

\begin{frame}[plain,t]
\begin{center}
\includegraphics[width=0.9\textwidth]{img/genetics.png}
\\[2ex]
Cause and effect of 1.6B genetic measurements, in Edward
\end{center}
\begin{tikzpicture}[remember picture,overlay]
  \node [xshift=-5.5cm, yshift=0.4cm, anchor=south west] at (current
  page.south east) {
\gray{\small [in preparation; fig from Gopalan+ 2017]}
  };
\end{tikzpicture}
\end{frame}

\begin{frame}[plain]
\begin{center}
\includegraphics[width=0.9\textwidth]{img/basketball.png}
\\[2ex]
Spatial analysis of 150,000 shots from 308 NBA players, in Edward
\end{center}
\begin{tikzpicture}[remember picture,overlay]
  \node [xshift=-2.5cm, yshift=0.4cm, anchor=south west] at (current
  page.south east) {
\gray{\small [Dieng+ 2017]}
  };
\end{tikzpicture}
\end{frame}

\begin{frame}
\frametitle{Probabilistic machine learning}

\begin{itemize}
\item A probabilistic model is a joint distribution of hidden
  variables $\mbz$ and observed variables $\mbx$,
  \begin{align*}
    p(\mbz, \mbx).
  \end{align*}
\item Inference about the unknowns is through the \textbf{posterior},
  the conditional distribution of the hidden variables given the observations
  \begin{align*}
    p(\mbz \g \mbx) = \frac{p(\mbz, \mbx)}{p(\mbx)}.
  \end{align*}
\item For most interesting models, the denominator is not tractable.
  We appeal to \textbf{approximate posterior inference}.
\end{itemize}

\end{frame}


\begin{frame}
\frametitle{Variational inference}

\begin{center}
  \includegraphics[width=0.75\textwidth]{img/vi_cartoon.pdf}
\end{center}

\vspace{0.1in}

\begin{itemize}
\item VI solves \textbf{inference} with \textbf{optimization}.
\item Posit a \textbf{variational family} of distributions over the
  latent variables,
  \begin{align*}
    q(\mbz; \mbnu)
  \end{align*}
\item Fit the \textbf{variational parameters} $\mbnu$ to be close (in
  KL) to the exact posterior.
\end{itemize}
\end{frame}

\begin{frame}
\frametitle{What is probabilistic programming?}
\textbf{Probabilistic programs reify models from mathematics to
physical objects.}
\begin{itemize}
\vspace{-2ex}
\item
Each model is equipped with memory (``bits'',
floating point, storage) and computation
(``flops'', scalability, communication).
% from Gauss and Fisher to Turing and Church
\end{itemize}
\textbf{Anything you do lives in the world of probabilistic programming.}
\begin{itemize}
\item
Any computable model.

\gray{ex.
graphical models; neural networks; SVMs; stochastic processes.
}
\item
Any computable inference algorithm.

\gray{ex.
automated inference;
model-specific algorithms;
inference within inference (learning to learn).
}
\item
Any computable application.

\gray{ex.
exploratory analysis;
object recognition;
code generation;
causality.
}
\end{itemize}
\end{frame}

\begin{frame}[plain,t]
\begin{center}
\end{center}
\begin{tikzpicture}[remember picture,overlay]
  \node [xshift=-0.5cm, yshift=0.0cm, anchor=north west] at (current
  page.north west) {
\includegraphics[width=1.25\textwidth]{img/wood2015nips.png}
  };
  \node [xshift=-3.25cm, yshift=0.4cm, anchor=south west] at (current
  page.south east) {
\gray{\small [fig. from Frank Wood]}
  };
\end{tikzpicture}
\end{frame}

\begin{frame}
\frametitle{George E.P. Box (1919 - 2013)}
\begin{columns}
\begin{column}{0.5\textwidth}
    \begin{center}
     \includegraphics[width=\columnwidth]{img/box.jpg}
     \end{center}
\end{column}
\begin{column}{0.5\textwidth}
An iterative process for science:
\\[1ex]
\begin{enumerate}
\item Build a model of the science
\\[1ex]
\item Infer the model given data
\\[1ex]
\item Criticize the model given data
\end{enumerate}
\end{column}
\end{columns}
\begin{tikzpicture}[remember picture,overlay]
  \node [xshift=-8.0cm, yshift=0.4cm, anchor=south west] at (current
  page.south east) {
\gray{\small [Box \& Hunter 1962, 1965; Box \& Hill 1967; Box 1976, 1980]}
  };
\end{tikzpicture}
\end{frame}

\begin{frame}
\frametitle{Box's Loop}
\begin{tikzpicture}[remember picture,overlay]
  \node [xshift=-1cm, yshift=-2.00cm, anchor=north west] at (current page.north west) {
\includegraphics[width=1.4\textwidth]{img/model_infer_criticize.png}
  };
  \node [xshift=3.4cm, yshift=-8.0cm, anchor=north west] at (current page.north west) {
Edward is a library designed around this loop.
  };
  \node [xshift=0cm, yshift=-5.50cm, anchor=north west] at (current page.north west) {
  };
  \node [xshift=-4.0cm, yshift=0.4cm, anchor=south west] at (current
  page.south east) {
\gray{\small [Box 1976, 1980; Blei 2014]}
  };
\end{tikzpicture}
\end{frame}

\begin{frame}
\begin{center}
\vspace{-3.5ex}
\includegraphics[width=1.0\textwidth]{img/github.png}
\\[-1ex]
\includegraphics[width=1.0\textwidth]{img/forum.png}
\\[-3ex]
\includegraphics[width=1.0\textwidth]{img/gitter.png}
\\[2ex]
\end{center}
\text{
We have an active community of several thousand users \& many
contributors.
}
\end{frame}

% \begin{frame}
% \frametitle{Who is Using Edward?}
% {\large Users}
% \begin{enumerate}
% \item
% Machine learning enthusiasts, data scientists, business analysts \\
% (\emph{ex. hierarchical GLMs, mixture models, MAP, MCMC, ...})
% \item
% Probabilistic graphical modeling community \\
% (\emph{ex. latent Dirichlet allocation, variational inference, Gibbs})
% \item
% Bayesian deep learning community \\
% (\emph{ex. deep generative models, Bayesian NNs, black box inference})
% \end{enumerate}

% {\large Developers}
% \begin{enumerate}
% \item
% David Blei's group
% \item
% Google Brain
% (\emph{design})
% \item
% Matt Hoffman (\emph{conjugacy}),
% Emily Fox's group
% (\emph{time series + SGMCMC}),
% Justin Bayer (\emph{stochastic RNNs}),
% John Pearson (\emph{neuroscience}),
% a few Master's/Ph.D. students.
% \item
% Collaboration continues to evolve. Contact us! (+visit the Forum)
% \end{enumerate}
% \end{frame}

\begin{frame}
\frametitle{Model}
Edward's language augments computational graphs with an abstraction
for random variables.
Each random variable $\mbx$ is associated to a tensor $\mbx^*$,
$\mbx^*\sim p(\mbx\g\theta^*)$.

\vspace{-1.0ex}
\includegraphics[height=0.20\textwidth]{img/random_variables.png}

Unlike \texttt{tf.Tensor}s, \texttt{ed.RandomVariable}s
carry an explicit density with methods
such as \texttt{log\_prob()} and \texttt{sample()}.

For implementation, we wrap all TensorFlow Distributions and call
\texttt{sample} to produce the associated tensor.
\begin{tikzpicture}[remember picture,overlay]
  \node [xshift=-2.2cm, yshift=0.4cm, anchor=south west] at (current
  page.south east) {
\gray{\small [Tran+ 2017]}
  };
\end{tikzpicture}
\end{frame}

\begin{frame}
\frametitle{Example: Beta-Bernoulli}
Consider a Beta-Bernoulli model,
\begin{align*}
p(\mbx, \theta) =
\operatorname{Beta}(\theta\g 1, 1)
\prod_{n=1}^{50} \operatorname{Bernoulli}(x_n\g \theta),
\end{align*}
where $\theta$
is a probability shared across 50 data points $\mbx\in\{0,1\}^{50}$.
\begin{center}
\vspace{-2ex}
\includegraphics[height=0.175\textwidth]{img/beta-bernoulli.png}
\end{center}
Fetching $\mbx$ from the graph generates a binary vector of $50$ elements.

All computation is represented on the graph, enabling us to leverage model structure during inference.
\end{frame}

\begin{frame}
\frametitle{Example: Variational Auto-Encoder for Binarized MNIST}
\begin{tikzpicture}[remember picture,overlay]
  \node [xshift=0.4cm, yshift=-0.8cm, anchor=north west] at (current page.north west) {
   \includegraphics[width=0.5\columnwidth]{img/vae_graphical_model_01.png}
  };
  \node [xshift=-5.75cm, yshift=0.4cm, anchor=south west] at (current
  page.south east) {
\gray{\small [Kingma \& Welling 2014; Rezende+ 2014]}
  };
\end{tikzpicture}
\end{frame}

\begin{frame}
\frametitle{Example: Variational Auto-Encoder for Binarized MNIST}
\begin{tikzpicture}[remember picture,overlay]
  \node [xshift=0.5cm, yshift=-1.0cm, anchor=north west] at (current page.north west) {
   \includegraphics[width=0.18\columnwidth]{img/vae_graphical_model.png}
  };
  \node [xshift=2.6cm, yshift=-1.7cm, anchor=north west] at (current page.north west) {
\includegraphics[height=0.14\textheight]{img/vae_decoder.png}
  };
  \node [xshift=2.6cm, yshift=-5.8cm, anchor=north west] at (current page.north west) {
\includegraphics[height=0.18\textheight]{img/vae_encoder.png}
  };
  \node [xshift=-5.75cm, yshift=0.4cm, anchor=south west] at (current
  page.south east) {
\gray{\small [Kingma \& Welling 2014; Rezende+ 2014]}
  };
\end{tikzpicture}
\end{frame}

\begin{frame}
\frametitle{Example: Variational Auto-Encoder for Binarized MNIST}
\vspace{10ex}
\begin{center}
\gray{\Large [Demo]} \\[3ex]
\end{center}
\end{frame}

\begin{frame}
\frametitle{Example: Bayesian neural network for classification}
\begin{center}
\includegraphics[height=0.3\textwidth]{img/bayesian_nn_graph.png}
\\[2.5ex]
\includegraphics[height=0.19\textwidth]{img/bayesian_nn_program.png}
\end{center}
\begin{tikzpicture}[remember picture,overlay]
  \node [xshift=-9.0cm, yshift=0.4cm, anchor=south west] at (current
  page.south east) {
\gray{\small [Denker+ 1987; MacKay 1992; Hinton \& Van Camp, 1993; Neal 1995]}
  };
\end{tikzpicture}
\end{frame}

\begin{frame}[plain]
\frametitle{Example: Gaussian process classification}
\begin{center}
\includegraphics[height=0.3\textwidth]{img/gp-classification.png}
\\[3.0ex]
\includegraphics[height=0.12\textwidth]{img/gp-classification-code.png}
\end{center}
\begin{tikzpicture}[remember picture,overlay]
  \node [xshift=-7.5cm, yshift=0.4cm, anchor=south west] at (current
  page.south east) {
\gray{\small [Rasmussen \& Williams, 2006; fig from Hensman+ 2013]}
  };
\end{tikzpicture}
\end{frame}

\begin{frame}
\frametitle{Inference}
Given
\begin{itemize}
\item Data $\mbx_{\text{train}}$.
\item
Model $p(\mbx, \mbz, \mbbeta)$ of
observed variables $\mbx$ and latent variables $\mbz, \mbbeta$.
\end{itemize}
Goal
\begin{itemize}
\item
Calculate posterior distribution
\begin{equation*}
p(\mbz, \mbbeta\mid\mbx_{\text{train}}) =
\frac{p(\mbx_{\text{train}}, \mbz, \mbbeta)}{\int
p(\mbx_{\text{train}}, \mbz, \mbbeta) \d\mbz\d\mbbeta}.
\end{equation*}
\end{itemize}
\vspace{2ex}
This is the key problem in Bayesian inference.
\begin{tikzpicture}[remember picture,overlay]
  \node [xshift=-4.5cm, yshift=0.4cm, anchor=south west] at (current
  page.south east) {
\small \url{edwardlib.org/tutorials}
  };
\end{tikzpicture}
\end{frame}

\begin{frame}
\frametitle{Inference}

\begin{center}
\includegraphics[width=0.7\textwidth]{img/inference-graph.png}
\end{center}

All \texttt{Inference} has (at least) two inputs: \\
\begin{enumerate}
\vspace{-3ex}
\item
\red{red} aligns latent variables and posterior approximations;
\item
\blue{blue} aligns observed variables and realizations.
\end{enumerate}

\begin{center}
\vspace{-2.0ex}
\includegraphics[height=0.05\textheight]{img/inference.png}
\end{center}

\texttt{Inference} has class methods to finely control the algorithm.
Edward is fast as handwritten TensorFlow at runtime.
\begin{tikzpicture}[remember picture,overlay]
  \node [xshift=-3.5cm, yshift=0.4cm, anchor=south west] at (current
  page.south east) {
\small \url{edwardlib.org/api}
  };
\end{tikzpicture}
\end{frame}

\begin{frame}
\frametitle{Inference}
Variational inference. It uses a variational model.
\begin{center}
\vspace{-2.0ex}
\includegraphics[height=0.18\textheight]{img/inference_variational.png}
\end{center}
Monte Carlo. It uses an Empirical approximation.
\begin{center}
\includegraphics[height=0.17\textheight]{img/inference_monte.png}
\end{center}

Conjugacy \& exact inference. It uses symbolic algebra on the graph.
\end{frame}

\begin{frame}
\frametitle{Inference: Composing Inference}
Core to Edward's design is that inference can be written as a collection of separate inference programs.

For example, here is variational EM.

\begin{center}
\includegraphics[height=0.275\textheight]{img/composing.png}
\end{center}

We can also write message passing algorithms, which work over a collection of local inference problems. This includes expectation propagation.
\begin{tikzpicture}[remember picture,overlay]
  \node [xshift=-9.2cm, yshift=0.4cm, anchor=south west] at (current
  page.south east) {
\gray{\small [Neal \& Hinton 1993; Minka 2001; Gelman+ 2017; Hasenclever+ 2015]}
  };
\end{tikzpicture}
\end{frame}

\begin{frame}
\frametitle{Non-Bayesian Methods: GANs}
GANs posit a generative process,
\begin{align*}
\mbepsilon &\sim \text{Normal}(0, 1) \\
\mbx &= G(\mbepsilon;\theta)
\end{align*}
for some generative network $G$.

Training uses a discriminative network $D$ via the
optimization problem
\begin{equation*}
\min_\theta \max_\phi
\mathbb{E}_{p^*(\mbx)} [ \log D(\mbx; \phi) ]
+ \mathbb{E}_{p(\mbx; \theta)} [ \log (1 - D(\mbx; \phi)) ]
\end{equation*}
The generator tries to generate samples indistinguishable from true data.

The discriminator tries to discriminate samples from the generator and
samples from the true data.

\begin{tikzpicture}[remember picture,overlay]
  \node [xshift=-3.0cm, yshift=0.4cm, anchor=south west] at (current
  page.south east) {
\gray{\small [Goodfellow+ 2014]}
  };
\end{tikzpicture}
\end{frame}

\begin{frame}
\frametitle{Example: Generative Adversarial Network for MNIST}
\vspace{10ex}
\begin{center}
\gray{\Large [Demo]} \\[3ex]
{\large \url{http://edwardlib.org/tutorials/gan}}
\end{center}
\end{frame}

\begin{frame}
\frametitle{Non-Bayesian Methods: GANs}
\begin{center}
\vspace{-2ex}
\includegraphics[width=1.0\textwidth]{img/gan_example.png}
\end{center}
\begin{tikzpicture}[remember picture,overlay]
  \node [xshift=-3.0cm, yshift=0.4cm, anchor=south west] at (current
  page.south east) {
\gray{\small [Goodfellow+ 2014]}
  };
\end{tikzpicture}
\end{frame}

\begin{frame}
\frametitle{Non-Bayesian Methods: GANs}
\begin{center}
\vspace{-2ex}
\includegraphics[width=1.1\textwidth]{img/ganw_example.png}
\end{center}
\begin{tikzpicture}[remember picture,overlay]
  \node [xshift=-4.5cm, yshift=0.4cm, anchor=south west] at (current
  page.south east) {
\gray{\small [Arjovsky+ 2017; Gulrajani+ 2017]}
  };
\end{tikzpicture}
\end{frame}

\begin{frame}
\begin{center}
{\Large\bf Current Work}
\end{center}
\end{frame}

\begin{frame}[t]
\frametitle{Dynamic Graphs}
\begin{center}
\vspace{-1.25ex}
\includegraphics[width=1.0\textwidth]{img/pyro.png}
\\[-8.75ex]
\includegraphics[width=1.0\textwidth]{img/probtorch.png}
\end{center}
\end{frame}

\begin{frame}
\frametitle{Distributions Backend}
\vspace{-3ex}
\begin{lstlisting}[language=python]
def pixelcnn_dist(params, x_shape=(32, 32, 3)):
  def _logit_func(features):
    # single autoregressive step on observed features
    logits = pixelcnn(features)
    return logits
  logit_template = tf.make_template("pixelcnn", _logit_func)
  make_dist = lambda x: tfd.Independent(tfd.Bernoulli(logit_template(x)))
  return tfd.Autoregressive(make_dist, tf.reduce_prod(x_shape)))

x = pixelcnn_dist()
loss = -tf.reduce_sum(x.log_prob(images))
train = tf.train.AdamOptimizer().minimize(loss)  # run for training
generate = x.sample()  # run for generation
\end{lstlisting}


\textbf{TensorFlow Distributions} consists of a large collection of
distributions. \texttt{Bijector} enable efficient, composable
manipulation of probability distributions.
% pixelcnn, autoregressive flows, reversible resnet

Pytorch PPLs are consolidating on a backend for distributions.
\begin{tikzpicture}[remember picture,overlay]
  \node [xshift=-2.25cm, yshift=0.4cm, anchor=south west] at (current
  page.south east) {
\gray{\small [Dillon+ 2017]}
  };
\end{tikzpicture}
\end{frame}

\begin{frame}
\frametitle{Distributed, Compiled, Accelerated Systems}
\begin{center}
\vspace{-1.25ex}
\includegraphics[width=0.8\textwidth]{img/tpu-pods.png}
\end{center}

\vspace{3ex}
Probabilistic programming over multiple machines.
XLA compiler optimization and TPUs.
% Turing-complete meta language for inference.
More flexible programmable inference.
\end{frame}

\begin{frame}
\frametitle{References}
\begin{center}
\includegraphics[width=0.3\textwidth]{img/edward.png}
\\
\large \url{edwardlib.org}
\end{center}
\vspace{1ex}

\begin{itemize}
\item
Edward: A library for probabilistic modeling, inference, and
criticism. \\
\gray{arXiv preprint arXiv:1610.09787, 2016.}
\item
Deep probabilistic programming. \\
\gray{International Conference on Learning Representations, 2017.}
\item
TensorFlow Distributions. \\
\gray{arXiv preprint arXiv:1711.10604, 2017.}
\end{itemize}
\end{frame}

\end{document}
