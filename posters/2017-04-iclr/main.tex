\documentclass[final]{beamer}

\usepackage[scale=1.24]{beamerposter}
\usetheme{confposter}
\usefonttheme{serif}

% \usepackage{etex}
\usepackage[T1]{fontenc}
\usepackage[bitstream-charter]{mathdesign}
\usepackage[scaled=0.92]{PTSans}
\renewcommand*\ttdefault{lmvtt}
\usepackage{exscale}

\usepackage[final,expansion=alltext]{microtype}
\usepackage[english]{babel}
\usepackage{ragged2e}

\usepackage{amsmath}
\usepackage{graphicx}                               % graphics
\usepackage{booktabs, array}                        % nice tables
\usepackage[algoruled]{algorithm2e}                 % pseudocode
\usepackage[numbers,sort]{natbib}                   % better bibliography
\usepackage{cleveref}                               % the best package ever
\usepackage[acronym,smallcaps,nowarn]{glossaries}   % second best package ever

% COLORS
\newcommand{\gray}[1]{\textcolor{black!60}{#1}}
\newcommand{\blu}[1]{\textcolor{jblue}{#1}}

\DeclareRobustCommand{\mb}[1]{\ensuremath{\boldsymbol{\mathbf{#1}}}}
\DeclareRobustCommand{\mbf}[1]{\ensuremath{\textbf{#1}}}
\DeclareMathOperator*{\argmax}{arg\,max}
\DeclareMathOperator*{\argmin}{arg\,min}


\newcommand{\KL}[2]{\ensuremath{\textrm{KL}\left(#1\;\|\;#2\right)}}

\renewcommand{\mid}{~\vert~}

\newcommand{\mbw}{\mb{w}}
\newcommand{\mbW}{\mb{W}}

\newcommand{\mbx}{\mb{x}}
\newcommand{\mbX}{\mbf{X}}

\newcommand{\mby}{\mb{y}}
\newcommand{\mbY}{\mbf{Y}}

\newcommand{\mbz}{\mb{z}}
\newcommand{\mbZ}{\mb{Z}}

\newcommand{\mbI}{\mbf{I}}
\newcommand{\mbone}{\mbf{1}}

\newcommand{\mbL}{\mbf{L}}

\newcommand{\mbtheta}{\mb{\theta}}
\newcommand{\mbTheta}{\mb{\Theta}}
\newcommand{\mbomega}{\mb{\omega}}
\newcommand{\mbOmega}{\mb{\Omega}}
\newcommand{\mbsigma}{\mb{\sigma}}
\newcommand{\mbSigma}{\mb{\Sigma}}
\newcommand{\mbphi}{\mb{\phi}}
\newcommand{\mbPhi}{\boldsymbol{\Phi}}
% \newcommand{\mblambda}{\mb{\lambda}}

\newcommand{\mbalpha}{\mb{\alpha}}
\newcommand{\mbbeta}{\mb{\beta}}
\newcommand{\mbgamma}{\mb{\gamma}}
\newcommand{\mbeta}{\mb{\eta}}
\newcommand{\mbmu}{\mb{\mu}}
\newcommand{\mbrho}{\mb{\rho}}
\newcommand{\mblambda}{\mb{\lambda}}
\newcommand{\mbzeta}{\mb{\zeta}}

\newcommand\dif{\mathop{}\!\mathrm{d}}
\newcommand{\diag}{\textrm{diag}}
\newcommand{\supp}{\textrm{supp}}

\newcommand{\E}{\mathbb{E}}
\newcommand{\bbH}{\mathbb{H}}
\newcommand{\Var}{\mathbb{V}\textrm{ar}}

\newcommand{\bbN}{\mathbb{N}}
\newcommand{\bbZ}{\mathbb{Z}}
\newcommand{\bbR}{\mathbb{R}}
\newcommand{\bbS}{\mathbb{S}}

\newcommand{\cL}{\mathcal{L}}

\newcommand{\cN}{\mathcal{N}}
\newcommand{\cT}{\mathcal{T}}
\newcommand{\Gam}{\textrm{Gam}}
\newcommand{\InvGam}{\textrm{InvGam}}

\newacronym{ELBO}{elbo}{evidence lower bound}
\newacronym{KL}{kl}{Kullback-Leibler}

\newacronym{DEF}{def}{deep exponential family}
\newacronym{DLGM}{dlgm}{deep latent Gaussian model}
\newacronym{DRAW}{draw}{Deep Recurrent Attentive Writer}

\newacronym{MF}{mf}{mean-field}
\newacronym{VGP}{vgp}{variational Gaussian process}
\newacronym{BBVI}{bbvi}{black box variational inference}
\newacronym{ADVI}{advi}{automatic differentiation variational inference}
\newacronym{NF}{nf}{normalizing flows}
\newacronym{CVI}{cvi}{copula variational inference}
\newacronym{VAE}{vae}{variational autoencoder}
\newacronym{IWAE}{iwae}{importance weighted autoencoder}
\newacronym{NVIL}{nvil}{neural variational inference}
\newacronym{MIXTURE}{mixture}{}
\newacronym{DSVI}{dsvi}{}

\newacronym{VI}{vi}{variational inference}
\newacronym{EP}{ep}{expectation propagation}

\newacronym{LS}{ls}{Langevin-Stein}
\newacronym{OPVI}{opvi}{operator variational inference}

\setbeamercolor{block title}{fg=jblue,bg=white} % Colors of the block titles
\setbeamercolor{block body}{fg=black,bg=white} % Colors of the body of blocks
\setbeamercolor{block alerted title}{fg=white,bg=dblue!70} % Colors of the highlighted block titles
\setbeamercolor{block alerted body}{fg=black,bg=dblue!10} % Colors of the body of highlighted blocks
% Many more colors are available for use in beamerthemeconfposter.sty

%-----------------------------------------------------------
% Define the column widths and overall poster size
% To set effective sepwid, onecolwid and twocolwid values, first choose how many columns you want and how much separation you want between columns
% In this template, the separation width chosen is 0.024 of the paper width and a 4-column layout
% onecolwid should therefore be (1-(# of columns+1)*sepwid)/# of columns e.g. (1-(4+1)*0.024)/4 = 0.22
% Set twocolwid to be (2*onecolwid)+sepwid = 0.464
% Set threecolwid to be (3*onecolwid)+2*sepwid = 0.708

\newlength{\sepwid}
\newlength{\onecolwid}
\newlength{\twocolwid}
\newlength{\threecolwid}
%\setlength{\paperwidth}{48in} % A0 width: 46.8in
%\setlength{\paperheight}{36in} % A0 height: 33.1in
\setlength{\sepwid}{0.024\paperwidth} % Separation width (white space) between columns
\setlength{\onecolwid}{0.301\paperwidth} % Width of one column
\setlength{\twocolwid}{0.626\paperwidth} % Width of two columns
\setlength{\threecolwid}{0.928\paperwidth} % Width of three columns
\setlength{\topmargin}{-0.5in} % Reduce the top margin size

\addtobeamertemplate{block end}{}{\vspace*{2ex}} % White space under blocks
\addtobeamertemplate{block alerted end}{}{\vspace*{2ex}} % White space under highlighted (alert) blocks

\setlength{\belowcaptionskip}{2ex} % White space under figures
\setlength\belowdisplayshortskip{2ex} % White space under equations
%-----------------------------------------------------------

\usepackage{tikz}
\usetikzlibrary{bayesnet}

\pgfdeclarelayer{edgelayer}
\pgfdeclarelayer{nodelayer}
\pgfsetlayers{edgelayer,nodelayer,main}

\definecolor{hexcolor0xbfbfbf}{rgb}{0.749,0.749,0.749}

\tikzset{>=latex}
\tikzstyle{none}   = [inner sep=0pt]
\tikzstyle{line}   = [-,
                      thick,
                      shorten <=1pt,
                      shorten >=1pt]
\tikzstyle{arrow}  = [->,
                      thick,
                      shorten <=1pt,
                      shorten >=1pt]
\tikzstyle{ardash} = [dashed,
                      ->,
                      thick,
                      shorten <=1pt,
                      shorten >=1pt]

\tikzstyle{empty}=[
                   circle,
                   opacity=0.0,
                   text opacity=1.0,
                   inner sep=0pt
                  ]

\tikzstyle{box}=[
                 rectangle,
                 fill=White,
                 thick,
                 draw=Black,
                 inner sep=7pt
                ]

\tikzstyle{filled}=[
                    circle,
                    thick,
                    fill=hexcolor0xbfbfbf,
                    draw=Black
                   ]

\tikzstyle{hollow}=[
                    circle,
                    thick,
                    fill=White,
                    draw=Black
                   ]

\tikzstyle{param}=[
                   rectangle,
                   fill=Black,
                   draw=Black,
                   inner sep=0pt,
                   minimum width=4pt,
                   minimum height=4pt
                  ]

\tikzstyle{paramhollow}=[
                         rectangle,
                         thick,
                         fill=White,
                         draw=Black,
                         inner sep=0pt,
                         minimum
                         width=4pt,
                         minimum height=4pt
                        ]

\usepackage{pgfplots}                               % PGFPLOTS baby!
\pgfplotsset{compat=newest}
\pgfplotsset{plot coordinates/math parser=false}
% \usepgfplotslibrary{statistics}


\title{Deep Probabilistic Programming with Edward}
\author{Dustin Tran\textsuperscript{\textdagger},
Matt Hoffman\textsuperscript{*}\textsuperscript{\ddag},
Kevin Murphy\textsuperscript{\ddag},
Eugene Brevdo\textsuperscript{\ddag},
Rif Saurous\textsuperscript{\ddag},
David Blei\textsuperscript{\textdagger}}
\institute{
\textsuperscript{\textdagger}Columbia University,
\textsuperscript{*}Adobe Research,
\textsuperscript{\ddag}Google
}

\begin{document}

\begin{frame}[t]
\begin{columns}[t]

\begin{column}{\sepwid}\end{column} % Empty spacer column

\begin{column}{\onecolwid}

\begin{alertblock}{Summary}
\begin{itemize}
  \item
Deep neural networks are popular in large part due to their

compositional nature. How do we do this for
probabilistic modeling?
  \item We describe
Edward, a Turing-complete \acrlong{PPL}.
\item Edward builds
two representations---random variables and
inference.
\item
For example, we show how to design rich variational models and \acrlongpl{GAN}.
\end{itemize}
\end{alertblock}

\begin{block}{Compositional Representations for Probabilistic Models}
\begin{itemize}
\item
We define random variables as the key compositional representation.
\item
They are class objects e.g. with log-density and sample methods.
\item
Each random variable $\mbx$ is associated to a
tensor $\mbx^*$ in the computational graph, which represents a single
sample $\mbx^*\sim p(\mbx)$.
\item
Mutable states represent enable conditioning sets to vary,
$p(\mby\g\mbx)$ and optimization of parameters, $p(\mbx; \theta)$.
\end{itemize}
\end{block}

\begin{block}{Compositional Representations for Inference}
\begin{itemize}
\item
Given data $\mbx_{\text{train}}$, inference aims to calculate the
posterior
$p(\mathbf{z}, \beta\mid \mathbf{x}_{\text{train}}; \mbtheta)$, where
$\mbtheta$ are any model parameters to estimate.
\item
In variational inference, the idea is to posit an approximating family
$q\in\mathcal{Q}$ and to find the closest member $q^*$.
We write it with mutable states representing its parameters,
where
$q(\beta;\mu,\sigma) = \operatorname{Normal}(\beta; \mu,\sigma)$,
$q(\mbz;\pi) = \operatorname{Categorical}(\mbz;\pi)$.
\vspace{2ex}
\hspace{-4.1em}
\includegraphics[height=5.25cm]{img/inference_variational.png}
\item
Specific variational algorithms inherit from
\texttt{VariationalInference} to define their own methods,
e.g., a
loss function and gradient.
\item
Monte Carlo approximates the posterior using samples.
We represent it where
the approximating family is an empirical distribution,
$q(\beta; \{\beta^{(t)}\}) = \frac{1}{T}\sum_{t=1}^T \delta(\beta,
\beta^{(t)})$,
$q(\mbz; \{\mbz^{(t)}\}) = \frac{1}{T}\sum_{t=1}^T \delta(\mbz,
\mbz^{(t)})$.

\vspace{1ex}
\includegraphics[height=5.25cm]{img/inference_monte.png}
\item
Monte Carlo
algorithms proceed by updating one sample $\beta^{(t)},\mbz^{(t)}$ at a time in the empirical
approximation.
Specific \glsunset{MC}\gls{MC} samplers determine the update rules.
\end{itemize}
\end{block}

\end{column}

\begin{column}{\sepwid}\end{column} % Empty spacer column

\begin{column}{\onecolwid}

\begin{block}{Example: Variational Auto-Encoder}
\begin{tabular}{cc}
\hspace{-2.15em}
\includegraphics{img/vae_graph.png}
&
\hspace{-0.5em}
\includegraphics[width=0.85\textwidth]{img/vae_code.png}
\end{tabular}
\vspace{-2ex}
\end{block}

\begin{block}{Example: Generative Adversarial Networks}
\begin{tabular}{cc}
\hspace{-1.5em}
\includegraphics{img/gan_graph.png}
&
\includegraphics[width=0.77\textwidth]{img/gan_code.png}
\end{tabular}
\vspace{-2ex}
\end{block}

\begin{block}{Example: Bayesian RNN with Variable Length}
\begin{tabular}{cc}
\hspace{-1.5em}
\includegraphics{img/bayesian_rnn_graph.png}
&
\includegraphics[width=0.80\textwidth]{img/bayesian_rnn_code.png}
\end{tabular}
\vspace{-2ex}
\end{block}

\begin{block}{Composing Inferences}
Core to Edward's design is that inference can be written as a collection
of separate inference programs. Below we demonstrate variational EM.
\vspace{1ex}

\begin{center}
\includegraphics{img/composing.png}
\end{center}

\end{block}

\end{column}

\begin{column}{\sepwid}\end{column} % Empty spacer column

\begin{column}{\onecolwid}

\begin{block}{Experiments: Recent Methods in Variational Inference}
\begin{table}[tb]
\centering
\begin{tabular}{lcc}
\toprule
Inference method & Negative log-likelihood
\\
\midrule
\Gls{VAE} \citep{kingma2014autoencoding} & $\le$ 88.2 \\
\gls{VAE} without analytic KL & $\le$ 89.4 \\
\gls{VAE} with analytic entropy & $\le$ 88.1 \\
\gls{VAE} with score function gradient & $\le$ 87.9 \\
Normalizing flows \citep{rezende2015variational} & $\le$ 85.8 \\
Hierarchical variational model \citep{ranganath2016hierarchical} & $\le$ 85.4 \\
Importance-weighted auto-encoders ($K=50$) \citep{burda2016importance}
& $\le$ 86.3 \\
\acrshort{HVM} with \acrshort{IWAE} objective ($K=5$)
& $\le$ 85.2 \\
R\'{e}nyi divergence ($\alpha=-1$) % \citep{li2016variational}
& $\le$ 140.5 \\
%\gls{GAN} objective \citep{goodfellow2014generative}& -- \\
\bottomrule
\end{tabular}
\end{table}

\vspace{1ex}
Inference methods for a probabilistic decoder on binarized
MNIST. The Edward \acrshort{PPL} enables fast experimentation with
many algorithms.
\end{block}

\begin{block}{Experiments: GPU-accelerated Hamiltonian Monte Carlo}
\begin{tabular}{cc}
\hspace{-1em}
\includegraphics{img/logistic_graph.png}
&
\includegraphics{img/logistic_code.png}
\end{tabular}
\vspace{1ex}

We apply Bayesian logistic regression to
Covertype ($N=581012$, $D=54$).

12-core Intel i7-5930K CPU at 3.50GHz, a NVIDIA Titan X (Maxwell) GPU.

We compare the runtime of HMC for 100 iterations (and same settings).
\vspace{1ex}

\begin{table}[tb]
\centering
\begin{tabular}{lr}
\toprule
Probabilistic programming system & Runtime (s)
\\
\midrule
Handwritten NumPy (1 CPU) & 534 \\
Stan (1 CPU) & 171 \\
PyMC3 (12 CPU) & 30.0 \\
\textbf{Edward (12 CPU)} & \textbf{8.2} \\
Handwritten TensorFlow (GPU) & 5.0 \\
\textbf{Edward (GPU)} & \textbf{4.9} (35x faster than Stan)\\
\bottomrule
\end{tabular}
\end{table}

\vspace{1ex}
Edward (GPU) is significantly faster than other systems. In addition,
Edward has no overhead: it is as fast as handwritten TensorFlow.
\end{block}

\vspace{-2ex}
\begin{block}{References}
\small{\bibliographystyle{apalike}
\bibliography{BIB}\vspace{0.75in}}
\end{block}

\end{column}

\begin{column}{\sepwid}\end{column} % Empty spacer column

\end{columns} % End of all the columns in the poster
\end{frame}   % End of the enclosing frame
\end{document}
