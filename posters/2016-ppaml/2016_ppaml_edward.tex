\documentclass{beamer}

\usepackage[scale=1.24,dvipsnames]{beamerposter}
\usetheme{confposter}
\usefonttheme{serif}

\usepackage{etex}
\usepackage[T1]{fontenc}
\usepackage[bitstream-charter]{mathdesign}
\usepackage[scaled=0.92]{PTSans}
\renewcommand*\ttdefault{lmvtt}
\usepackage{exscale}

\usepackage[final,expansion=alltext]{microtype}
\usepackage[english]{babel}
\usepackage{ragged2e}

\usepackage{amsmath}
\usepackage{graphicx}                               % graphics
\usepackage{booktabs, array}                        % nice tables
\usepackage[algoruled]{algorithm2e}                 % pseudocode
\usepackage[numbers,sort]{natbib}                   % better bibliography
\usepackage{cleveref}                               % the best package ever
\usepackage[acronym,smallcaps,nowarn]{glossaries}   % second best package ever

% COLORS
\newcommand{\gray}[1]{\textcolor{black!60}{#1}}
\newcommand{\blu}[1]{\textcolor{jblue}{#1}}

\input{preamble/preamble_size_portrait_3col}
\usepackage{tikz}
\usetikzlibrary{bayesnet}

\pgfdeclarelayer{edgelayer}
\pgfdeclarelayer{nodelayer}
\pgfsetlayers{edgelayer,nodelayer,main}

\definecolor{hexcolor0xbfbfbf}{rgb}{0.749,0.749,0.749}

\tikzset{>=latex}
\tikzstyle{none}   = [inner sep=0pt]
\tikzstyle{line}   = [-,
                      thick,
                      shorten <=1pt,
                      shorten >=1pt]
\tikzstyle{arrow}  = [->,
                      thick,
                      shorten <=1pt,
                      shorten >=1pt]
\tikzstyle{ardash} = [dashed,
                      ->,
                      thick,
                      shorten <=1pt,
                      shorten >=1pt]

\tikzstyle{empty}=[
                   circle,
                   opacity=0.0,
                   text opacity=1.0,
                   inner sep=0pt
                  ]

\tikzstyle{box}=[
                 rectangle,
                 fill=White,
                 thick,
                 draw=Black,
                 inner sep=7pt
                ]

\tikzstyle{filled}=[
                    circle,
                    thick,
                    fill=hexcolor0xbfbfbf,
                    draw=Black
                   ]

\tikzstyle{hollow}=[
                    circle,
                    thick,
                    fill=White,
                    draw=Black
                   ]

\tikzstyle{param}=[
                   rectangle,
                   fill=Black,
                   draw=Black,
                   inner sep=0pt,
                   minimum width=4pt,
                   minimum height=4pt
                  ]

\tikzstyle{paramhollow}=[
                         rectangle,
                         thick,
                         fill=White,
                         draw=Black,
                         inner sep=0pt,
                         minimum
                         width=4pt,
                         minimum height=4pt
                        ]

\usepackage{pgfplots}                               % PGFPLOTS baby!
\pgfplotsset{compat=newest}
\pgfplotsset{plot coordinates/math parser=false}
% \usepgfplotslibrary{statistics}

\DeclareRobustCommand{\mb}[1]{\ensuremath{\boldsymbol{\mathbf{#1}}}}
\DeclareRobustCommand{\mbf}[1]{\ensuremath{\textbf{#1}}}
\DeclareMathOperator*{\argmax}{arg\,max}
\DeclareMathOperator*{\argmin}{arg\,min}


\newcommand{\KL}[2]{\ensuremath{\textrm{KL}\left(#1\;\|\;#2\right)}}

\renewcommand{\mid}{~\vert~}

\newcommand{\mbw}{\mb{w}}
\newcommand{\mbW}{\mb{W}}

\newcommand{\mbx}{\mb{x}}
\newcommand{\mbX}{\mbf{X}}

\newcommand{\mby}{\mb{y}}
\newcommand{\mbY}{\mbf{Y}}

\newcommand{\mbz}{\mb{z}}
\newcommand{\mbZ}{\mb{Z}}

\newcommand{\mbI}{\mbf{I}}
\newcommand{\mbone}{\mbf{1}}

\newcommand{\mbL}{\mbf{L}}

\newcommand{\mbtheta}{\mb{\theta}}
\newcommand{\mbTheta}{\mb{\Theta}}
\newcommand{\mbomega}{\mb{\omega}}
\newcommand{\mbOmega}{\mb{\Omega}}
\newcommand{\mbsigma}{\mb{\sigma}}
\newcommand{\mbSigma}{\mb{\Sigma}}
\newcommand{\mbphi}{\mb{\phi}}
\newcommand{\mbPhi}{\boldsymbol{\Phi}}
% \newcommand{\mblambda}{\mb{\lambda}}

\newcommand{\mbalpha}{\mb{\alpha}}
\newcommand{\mbbeta}{\mb{\beta}}
\newcommand{\mbgamma}{\mb{\gamma}}
\newcommand{\mbeta}{\mb{\eta}}
\newcommand{\mbmu}{\mb{\mu}}
\newcommand{\mbrho}{\mb{\rho}}
\newcommand{\mblambda}{\mb{\lambda}}
\newcommand{\mbzeta}{\mb{\zeta}}

\newcommand\dif{\mathop{}\!\mathrm{d}}
\newcommand{\diag}{\textrm{diag}}
\newcommand{\supp}{\textrm{supp}}

\newcommand{\E}{\mathbb{E}}
\newcommand{\bbH}{\mathbb{H}}
\newcommand{\Var}{\mathbb{V}\textrm{ar}}

\newcommand{\bbN}{\mathbb{N}}
\newcommand{\bbZ}{\mathbb{Z}}
\newcommand{\bbR}{\mathbb{R}}
\newcommand{\bbS}{\mathbb{S}}

\newcommand{\cL}{\mathcal{L}}

\newcommand{\cN}{\mathcal{N}}
\newcommand{\cT}{\mathcal{T}}
\newcommand{\Gam}{\textrm{Gam}}
\newcommand{\InvGam}{\textrm{InvGam}}

% \newacronym{ELBO}{elbo}{evidence lower bound}
\newacronym{KL}{kl}{Kullback-Leibler}

\newacronym{DEF}{def}{deep exponential family}
\newacronym{DLGM}{dlgm}{deep latent Gaussian model}
\newacronym{DRAW}{draw}{Deep Recurrent Attentive Writer}

\newacronym{MF}{mf}{mean-field}
\newacronym{VGP}{vgp}{variational Gaussian process}
\newacronym{BBVI}{bbvi}{black box variational inference}
\newacronym{ADVI}{advi}{automatic differentiation variational inference}
\newacronym{NF}{nf}{normalizing flows}
\newacronym{CVI}{cvi}{copula variational inference}
\newacronym{VAE}{vae}{variational autoencoder}
\newacronym{IWAE}{iwae}{importance weighted autoencoder}
\newacronym{NVIL}{nvil}{neural variational inference}
\newacronym{MIXTURE}{mixture}{}
\newacronym{DSVI}{dsvi}{}

\newacronym{VI}{vi}{variational inference}
\newacronym{EP}{ep}{expectation propagation}

\newacronym{LS}{ls}{Langevin-Stein}
\newacronym{OPVI}{opvi}{operator variational inference}

% % LISTINGS DEFINTIONS
\usepackage{listings}
\lstset{language=C++,
  keywordstyle=\color{MidnightBlue}\bfseries,
  keywordstyle=[2]\color{BrickRed}\bfseries,
  keywordstyle=[3]\color{Violet},
  morekeywords={vector, real, in},
  keywords=[2]{data, parameters, model, transformed},
  keywords=[3]{lower, upper, normal, bernoulli_logit}
}
\lstdefinestyle{mystyle}{
    commentstyle=\color{ForestGreen},
    numberstyle=\tiny\color{black!60},
    stringstyle=\color{BrickRed},
    basicstyle=\ttfamily\small,
    breakatwhitespace=false,
    breaklines=true,
    captionpos=b,
    keepspaces=true,
    numbers=none,
    numbersep=5pt,
    showspaces=false,
    showstringspaces=false,
    showtabs=false,
    tabsize=2
}
\lstset{style=mystyle}


\title{Edward: a library for probabilistic modeling, inference, and criticism.}
\author{
Dustin Tran, David Blei, Alp Kucukelbir, Adji Dieng, Maja Rudolph, and Dawen Liang
}
\institute{
Columbia University
}



\begin{document}

\begin{frame}[t] 
\begin{columns}[t,totalwidth=10in]  % HACK?!?!?

\begin{column}{\sepwid}\end{column} % Empty spacer column

%--COL 1------------------------------------------------------------------------
\begin{column}{\onecolwid} 

\begin{alertblock}{Summary}
\begin{itemize}
  \item Edward is a library for probabilistic modeling, inference, and
  criticism \citep{tran2016edward}.
  \item Edward supports probability models $p(\mbx,\mbz)$.
  \item Edward leverages black box variational inference.
  \item Edward enables model and inference criticism.
  \item Edward is a Python/TensorFlow project.\newline\newline
  \textcolor{Fuchsia}{\texttt{https://github.com/blei-lab/edward}}\newline
\end{itemize}
\end{alertblock}

\begin{block}{Goals}
\begin{center}
\includegraphics[width=9in]{img/edward_venn.pdf}
\end{center}
\begin{itemize}
  \item Edward is an open-source research library for probabilistic programming
  research.\newline
  % \item Edward is named after the innovative statistician George Edward Pelham
  % Box.\newline
  \item Edward follows Box's philosophy of statistics and machine learning
  \citep{box1976science}
\end{itemize}
\begin{enumerate}
  \item Build a probabilistic model of the process
  \item Reason about the process given model and data
  \item Criticize the model, revise and repeat
\end{enumerate}
\end{block}

\begin{block}{Features}
\begin{itemize}
  \item Edward supports the following modeling languages\newline
  \begin{itemize}
    \item TensorFlow (with neural network composition via Keras, Pretty
    Tensor, or TensorFlow-Slim)
    \item Stan
    \item PyMC3
    \item Python through Numpy/Scipy\newline
  \end{itemize}
  \item Edward implements black box inference through variational inference
  \newline
  \begin{itemize}
    \item black box variational inference \citep{ranganath2014black}
    \item data-level stochastic variational inference \citep{hoffman2013stochastic}
    \item variational auto-encoders \citep{kingma2013auto}
    \item delta function / MAP approximation
    \item Laplace approximation\newline
  \end{itemize}
  \item Edward supports model and inference criticism\newline
  \begin{itemize}
    \item posterior predictive checks \citep{gelman1996posterior}
    \item a library of evaluation metrics
  \end{itemize}
\end{itemize}
\end{block}

\begin{block}{Backend}
\begin{itemize}
  \item Edward is built on top of TensorFlow\newline
  \begin{itemize}
    \item computation graphs
    \item parallelization / GPU support
    \item automatic differentiation
    \item optimization algorithms\newline
  \end{itemize}
  \item Edward implements its own math/probability library.
\end{itemize}
\end{block}

\end{column} 
%-------------------------------------------------------------------------------

\begin{column}{\sepwid}\end{column} % Empty spacer column

%--COL 2------------------------------------------------------------------------
\begin{column}{\onecolwid} 

\begin{block}{Scope}
\begin{center}
\includegraphics[width=9in]{img/prob_venn.pdf}
\end{center}
\begin{itemize}
  \item Edward focuses on probability models $p(\mbx,\mbz)$.
  \item Edward supports models with\newline
  \begin{itemize}
    \item large data $\mbx$
    \item continuous or discrete latent variables $\mbz$
    \item complex structures: e.g.~hierarchical models, neural networks,
    deep exponential families.
  \end{itemize}
  \item The goal is to infer the posterior $p(\mbz\mid\mbx)$.
\end{itemize}
\end{block}

\begin{block}{Design}
\blu{\large\textbf{Data}}\\
Edward \texttt{Data} objects are containers. A \texttt{Data} object has structure 
(dimensions); these must match a probability model during inference. A \texttt
{Data} object may optionally implement a custom subsampling routing; the default
is to subsample along the first dimension.

\vspace*{0.5in}

\blu{\large\textbf{Models}}\\
Edward has two types of \texttt{Models}:
\begin{enumerate}
  \item Probability models of data and latent variables
  \item Variational models of latent variables
\end{enumerate}

Probability models must implement
\begin{center}
\texttt{log\_prob(self, x, theta)}
\end{center}

Variational models must implement
\begin{center}
\texttt{sample(self, size=1)} and \texttt{entropy(self)}
\end{center}

\vspace*{0.5in}

\blu{\large\textbf{Inference}}\\
Edward supports many forms of variational inference. One form matches the
variational model to the posterior $p(\mbz\mid\mbx)$ by maximizing
\begin{align}
  \textsc{elbo} &= 
  \E_{q(\mbz\;;\;\mblambda)}
  \left[
  \log p(\mbx,\mbz)
  -
  \log q(\mbz\;;\;\mblambda)
  \right].
  \label{eq:elbo}
\end{align}
Edward solves this optimization using automatic differentiation and stochastic
gradient methods.

\vspace*{0.5in}

\blu{\large\textbf{Criticism}}\\
Edward provides building blocks for criticizing both model and inference.
An example is a pipeline for simulating new datasets using samples from the
variational approximation; this enables posterior predictive checks.

\end{block}

\begin{block}{Smart Inference}
Solving \Cref{eq:elbo} is the main computational task of inference. While
automatic differentiation helps by avoiding manual derivations of gradients, the
expectation in the objective function poses a greater challenge.\newline

There are two approaches to computing gradients of \cref{eq:elbo}
\begin{enumerate}
  \item score function estimator \hfill \gray{(more general)}
  \item reparameterization estimator \hfill \gray{(less noisy)}\newline
\end{enumerate}

Edward prefers reparameterization, if the variational model admits it.
Otherwise, it defaults to the score function estimator.\newline

Edward also prefers an analytic (closed-form) entropy term
$\E_{q(\mbz\;;\;\mblambda)} \left[-\log q(\mbz\;;\;\mblambda)\right]$, 
if the variational model admits it.
\end{block}

\end{column} 
%-------------------------------------------------------------------------------

\begin{column}{\sepwid}\end{column} % Empty spacer column

%--COL 4------------------------------------------------------------------------
\begin{column}{\onecolwid} 

\begin{block}{Example \emph{(abridged)}}
\begin{figure}[h]
\lstinputlisting{img/bernoulli.py}
\end{figure}
\end{block}

\begin{block}{Next Steps}
\blu{\large\textbf{Data}}\\
\begin{itemize}
  \item distributed / data in the cloud
  \item streaming data
\end{itemize}

\vspace*{0.5in}

\blu{\large\textbf{Models}}\\
\begin{itemize}
  \item a new modeling language
\end{itemize}

\vspace*{0.5in}

\blu{\large\textbf{Inference}}\\
\begin{itemize}
  \item subsampling of latent variables
  \item amortized variational inference
  \item marginal maximum likelihood
  \item alternative divergence measures
\end{itemize}

\vspace*{0.5in}

\blu{\large\textbf{Criticism}}\\
\begin{itemize}
  \item library of built-in predictive checks
\end{itemize}

\end{block}

\begin{block}{License}
Edward is open-source licensed under the Apache License, version 2.0.
\end{block}

\begin{block}{References}
\small{
\bibliographystyle{unsrt}
\bibliography{BIB}
}
\end{block}

\end{column} 
%-------------------------------------------------------------------------------

\end{columns} % End of all the columns in the poster
\end{frame}   % End of the enclosing frame
\end{document}
